%%%%%%%%%%%%%%
% Evaluation %
%%%%%%%%%%%%%%

\chapter{Evaluation}
\label{chapter:evaluation}

\section{Erfüllung der Kriterien}

\begin{enumerate}[label={K.\arabic*}]

\item
\label{eval:gui}
\textbf{GUI}

\item
\label{eval:interactivity}
\textbf{Interaktivität}

\item
\label{eval:immediate-feedback}
\textbf{Unmittelbares Feedback}

\item
\label{eval:editing-support}
\textbf{Förderung des Prozesses der Diagramm-Erstellung}

\item
\label{eval:mental-map}
\textbf{Erhaltung des mentalen Modells}

\item
\label{eval:focus-on-the-content}
\textbf{Förderung der Konzentration auf den Inhalt}

\item
\label{eval:aesthetics-criteria}
\textbf{Berücksichtigung der ästhetischen Prinzipien}

\item
\label{eval:syntax-and-semantics}
\textbf{Berücksichtigung der Syntax und Semantik}

\item
\label{eval:user-friendly}
\textbf{Benutzerfreundlichkeit}

\end{enumerate}

% Kriterien aus 4.1 + ästhetische Prinzipien in K.7

%Incremental editing: Small diagrams appear to be manually drawn faster than larger diagrams, but as requirements change and diagrams have to be adjusted, especially if elements in the center have to be inserted or deleted, layout algorithms help saving time. Of course, manual editing does not scale with the size of the diagram [Protsko et al. 1991; Sugiyama 2002; Eiglsperger 2003; Spinellis 2003].

% Problem, wenn viele Klassen im Diagramm

\section{Nutzerstudie}

\subsection{Aufbau}

% Hardware: Mac
% Software: Mac OS X, Prototyp, Bildschirmaufnahme mit Screeny (Link)

% Ablauf der Session (siehe Mindmap)
% dem Probanden wurde eine kurze Einführung über Klassendiagramme gegeben (Klassen, Vererbung) anhand von Papier
% Einführung Prototyp, da Anlernen notwendig
% Beschreibung der Aufgaben (am baumbasierten Layout)

\subsection{Auswertung}

% Probandengruppe beschreiben (alter, Studenten, Erfahrungen...)

% Beobachtungen
% Ergebnisse und Auswertung des Fragebogens
% Interpretation der Beobachtungen und Ergebnisse

\section{Zusammenfassung}
