%%%%%%%%%%%%%%
% Evaluation %
%%%%%%%%%%%%%%

\chapter{Evaluation}
\label{chapter:evaluation}

\section{Erfüllung der Kriterien}

\begin{enumerate}[label={K.\arabic*}]

\item
\label{eval:gui}
\textbf{GUI}

\item
\label{eval:interactivity}
\textbf{Interaktivität}

\item
\label{eval:immediate-feedback}
\textbf{Unmittelbares Feedback}

\item
\label{eval:editing-support}
\textbf{Förderung des Prozesses der Diagramm-Erstellung}

\item
\label{eval:mental-map}
\textbf{Erhaltung des mentalen Modells}

\item
\label{eval:focus-on-the-content}
\textbf{Förderung der Konzentration auf den Inhalt}

\item
\label{eval:aesthetics-criteria}
\textbf{Berücksichtigung der ästhetischen Prinzipien}

% ästhetische Prinzipien in K.7

\item
\label{eval:syntax-and-semantics}
\textbf{Berücksichtigung der Syntax und Semantik}

\item
\label{eval:user-friendly}
\textbf{Benutzerfreundlichkeit}

\end{enumerate}

% kleine Diagramme sind schnell zu zeichnen, manuelle Bearbeitung skaliert nicht mit der Größe des Diagramms \cite{Eichelberger05Aesthetics}
% -> Problem, wenn viele Klassen im Diagramm

\section{Nutzerstudie}

\subsection{Aufbau}

% Hardware
% - Mac
% - Trackpad, Apple Magic Mouse oder klassische optische Maus zur Auswahl
% Software
% - Mac OS X
% - Prototyp
% - Bildschirmaufnahme mit Screeny (http://screenyapp.com)

% Ablauf der Session (siehe Mindmap)
% dem Probanden wurde eine kurze Einführung über Klassendiagramme gegeben und Begriffe wie Klasse, Vererbung, Oberklasse und Unterklasse erklärt
% Einführung Prototyp, da Anlernen notwendig
% Beschreibung der möglichen Aktionen
% Aufgaben (am baumbasierten Layout)
% Verweis auf Material im Anhang

% Probandengruppe beschreiben (Alter, Studenten, Erfahrungen...)

\subsection{Auswertung}

% Beobachtungen
% Ergebnisse und Auswertung des Fragebogens
% Interpretation der Beobachtungen und Ergebnisse

% Fehlende Aktionen im Prototyp

\section{Zusammenfassung}
