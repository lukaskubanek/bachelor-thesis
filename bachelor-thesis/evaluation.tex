%%%%%%%%%%%%%%
% Evaluation %
%%%%%%%%%%%%%%

\chapter{Evaluation}
\label{chapter:evaluation}

\section{Erfüllung der Kriterien}

% Kriterien aus Abschnitt \ref{sec:criteria}

\begin{enumerate}[label={K.\arabic*}]

\item
\label{eval:gui}
\textbf{GUI}

\item
\label{eval:interactivity}
\textbf{Interaktivität}

\item
\label{eval:immediate-feedback}
\textbf{Unmittelbares Feedback}

\item
\label{eval:editing-support}
\textbf{Förderung des Prozesses der Diagramm-Erstellung}

\item
\label{eval:mental-map}
\textbf{Erhaltung des mentalen Modells}

\item
\label{eval:focus-on-the-content}
\textbf{Förderung der Konzentration auf den Inhalt}

\item
\label{eval:aesthetics-criteria}
\textbf{Berücksichtigung der ästhetischen Prinzipien}

% ästhetische Prinzipien in K.7

\item
\label{eval:syntax-and-semantics}
\textbf{Berücksichtigung der Syntax und Semantik}

\item
\label{eval:user-friendly}
\textbf{Benutzerfreundlichkeit}

\end{enumerate}

% kleine Diagramme sind schnell zu zeichnen, manuelle Bearbeitung skaliert nicht mit der Größe des Diagramms \cite{Eichelberger05Aesthetics}
% -> Problem, wenn viele Klassen im Diagramm

\section{Nutzerstudie}

Um das umgesetzte Konzept zu validieren, wurde im Rahmen der Bachelor-Arbeit eine Nutzerstudie durchgeführt, die den Gegenstand für die folgenden Abschnitte bildet. Zunächst werden in Abschnitt \ref{subsec:user-study-setup} der Aufbau und die Durchführung im Detail erläutert. Anschließend folgt in Abschnitt \ref{subsec:user-study-evaluation} eine Auswertung der Ergebnisse.

\subsection{Aufbau und Durchführung}
\label{subsec:user-study-setup}

Die Nutzerstudie wurde in individuelle Sitzungen aufgeteilt. Jede Sitzung hat mit einer kurzen Beschreibung des Themas der Bachelor-Arbeit angefangen. Danach wurde der Ablauf der Sitzung vorgestellt und der Teilnehmer wurde um das Ausfüllen des ersten Teils des vorgefertigten Fragebogens (siehe Abschnitt \ref{sec:user-study-material-questionnaire}) gebeten, in dem allgemeine Informationen und Vorkenntnisse abgefragt wurden. Nach Bedarf wurde anschließend eine kurze Einführung über Klassendiagramme gegeben, um die Notation und die wichtigsten Begriffe wie Klasse, Vererbung, Oberklasse und Unterklasse zu erklären.

Die eigentlichen Tests des entwickelten Prototyps (siehe Kapitel \ref{chapter:prototype}) wurden an einem \textit{MacBook Pro} durchgeführt. Als Eingabegerät stand den Teilnehmern das in dem Notebook eingebaute Trackpad, eine \textit{Apple Magic Mouse}\footnote{\url{https://www.apple.com/magicmouse}} und eine klassische optische Maus zur Verfügung. Um potenzielle Probleme mit der Eingabe zu vermeiden, konnten die Teilnehmer frei das passende Eingabegerät wählen.

Als Nächstes wurde eine kurze Einführung zu dem Prototyp gegeben. Die unterstützten Aktionen wie das Erstellen bzw. Umbenennen einer Klasse, Erstellen einer Vererbungsrelation und Löschen des Diagramms wurden beschrieben und im Prototyp demonstriert. Insbesondere wurden die zwei Möglichkeiten zum Erstellen der Vererbungsrelation gezeigt, nämlich durch das Anklicken des Icons in der Sidebar und durch das Drücken der der Taste \texttt{CTRL} (siehe Abschnitt \ref{subsec:supported-actions}). Nach dieser Einführung durfte der Teilnehmer den Prototyp für ca. zwei Minuten ausprobieren.

Der eigentliche Test bestand darin, vier verschiedene Aufgaben zur Modellierung von Klassendiagrammen mit Vererbungshierarchien zu erfüllen. Alle Aufgaben haben das baumbasierte Layout benötigt, so dass die Lay\-out-Engine nicht geändert werden musste. Dem Teilnehmer wurden nacheinander Blätter mit den Aufgabenstellungen vorgelegt, die durch ihn gelöst wurden. Zum einen handelte sich um visuelle Aufgaben, in den der Teilnehmer ein bestehendes Diagramm rekonstruieren sollte. Zum anderen hatten die Aufgaben die Form einer textuellen Beschreibung. Alle Aufgabenblätter sind im Anhang unter Abschnitt \ref{sec:user-study-material-task-description} abgebildet.

Die Teilnehmer wurden gebeten, während der Erfüllung der Aufgaben ihr Vorgehen laut zu beschreiben. Dabei wurden die wichtigsten Beobachtungen notiert. Weiterhin wurde der Bildschirm aufgenommen, um eventuelle weitere Analysen zu ermöglichen. Die Videodateien sind auf der beigelegten DVD zu finden (für das Dateienverzeichnis siehe Abschnitt \ref{sec:files-user-study}).

Im Anschluss wurde das Konzept durch den Teilnehmer bewertet, indem der verbleibende Teil des Fragebogens (siehe Abschnitt \ref{sec:user-study-material-questionnaire}) ausgefüllt wurde. Obwohl die Teilnehmer in den letzten Fragen des Fragebogens eine Möglichkeit hatten, ein allgemeines Feedback zu geben, wurde in einzelnen Fällen eine Diskussion bevorzugt.

\subsection{Auswertung}
\label{subsec:user-study-evaluation}

An der Nutzerstudie haben insgesamt 7 Testpersonen teilgenommen und beide Geschlechter wurden ungefähr gleichmäßig vertreten (4 männliche und 3 weibliche Teilnehmer). Großteils handelte sich um Studenten und wissenschaftliche Arbeiter zwischen 21 und 34 Jahren. Des Weiteren waren 6 von 7 Teilnehmern im Bereich der Informatik tätig und waren daher mit der Notation von Klassendiagrammen vertraut.

Vor dem Test wurden die Teilnehmer in dem Fragebogen (siehe Abschnitt \ref{sec:user-study-material-questionnaire}) befragt, welche Werkzeuge sie zum Erstellen von Diagrammen benutzen und in welchen Maß. Die kompletten Ergebnisse sind in Form von Diagrammen in Abbildung \ref{fig:used-tools-charts} veranschaulicht. Es stellt sich heraus, dass die Teilnehmer Stift und Papier zum Zeichnen von Diagrammen am häufigsten verwenden. Das kann dadurch begründet werden, dass dies oft die einfachste Methode ist \cite{Ambler02Agile}. Die Softwarewerkzeuge werden in kleinerem Maß eingesetzt, dennoch ist deren Nutzung nicht vernachlässigbar. Überwiegend werden die Diagramme in Präsentationsprogrammen gezeichnet, es werden aber auch Diagramm-Programme und UML-Editoren zum Einsatz gebracht. Dagegen sind die Ergebnisse der Nutzung von Online-Tools eher gering. 

\begin{figure}[hbt]
\newcommand{\legendscale}{0.9}
\newcommand{\subfigurewidth}{0.3\linewidth}
\newcommand{\graphicsscale}{0.75}    
\centering
\begin{minipage}{.15\textwidth}
    \includegraphics[scale=\legendscale]{resources/used-tools-charts-legend}
\end{minipage}
\begin{minipage}{.65\textwidth}
    \centering
    \begin{subfigure}[t]{\subfigurewidth}
        \centering
        \includegraphics[scale=\graphicsscale]{resources/used-tools-charts-a}
        \caption{Diagramm-Programme}
        \label{fig:used-tools-charts-a}
    \end{subfigure}
    \begin{subfigure}[t]{\subfigurewidth}
        \centering
        \includegraphics[scale=\graphicsscale]{resources/used-tools-charts-b}
        \caption{Präsentationsprogramme}
        \label{fig:used-tools-charts-b}
    \end{subfigure}
    \begin{subfigure}[t]{\subfigurewidth}
        \centering
        \includegraphics[scale=\graphicsscale]{resources/used-tools-charts-c}
        \caption{Online-Tools}
        \label{fig:used-tools-charts-c}
    \end{subfigure}
    \begin{subfigure}[t]{\subfigurewidth}
        \centering
        \includegraphics[scale=\graphicsscale]{resources/used-tools-charts-d}
        \caption{UML-Editoren}
        \label{fig:used-tools-charts-d}
    \end{subfigure}
    \begin{subfigure}[t]{\subfigurewidth}
        \centering
        \includegraphics[scale=\graphicsscale]{resources/used-tools-charts-e}
        \caption{Stift und Papier}
        \label{fig:used-tools-charts-e}
    \end{subfigure}
    \begin{subfigure}[t]{\subfigurewidth}
        \centering
        \includegraphics[scale=\graphicsscale]{resources/used-tools-charts-f}
        \caption{Whiteboard}
        \label{fig:used-tools-charts-f}
    \end{subfigure}
\end{minipage}
\caption{Auswertung der Verwendung von Tools zum Erstellen von Diagrammen}
\label{fig:used-tools-charts}
\end{figure}

Laut der Angaben der Teilnehmer konnten die Aufgaben in allen Fällen erfolgreich erfüllt werden (siehe Abbildung \ref{fig:evaluation-charts-a}). Weiterhin wurden die Aufgaben als nicht anstrengend eingeschätzt (siehe Abbildung \ref{fig:evaluation-charts-b}). Dies deutet darauf hin, dass für die Nutzerstudie angemessene Aufgaben gewählt wurden.

Obwohl es eine Einführung zu dem Prototypen gab (siehe Abschnitt \ref{subsec:user-study-setup}), wurden die Teilnehmer gefragt, ob sie die Aufgaben auch ohne diese Einführung erledigen könnten. Dies wurde zum großen Teil bestätigt, dennoch waren die Ergebnisse nicht ganz eindeutig (siehe Abbildung \ref{fig:evaluation-charts-c}). Bei der Vorbereitung des Aufbaus wurde überlegt, die Einführung wegzulassen. Da es sich aber bei den Softwaremodellierungswerkzeugen um ein Szenario handelt, in dem in der Regel das Anlernen der Bedienung und der Funktionen notwendig ist, wurde die Variante mit der Einführung bevorzugt.

\begin{figure}[!ht]

\newcommand{\legendscale}{0.9}
\newcommand{\subfigurewidth}{0.2\linewidth}
\newcommand{\graphicsscale}{0.75}    
\centering

% legend
\begin{subfigure}[t]{\textwidth}
    \centering
    \includegraphics[scale=\legendscale]{resources/evaluation-charts-legend}
    \vspace{0.5cm}
\end{subfigure}

% charts
\foreach \i in {a,...,l}{%
\begin{subfigure}[t]{\subfigurewidth}
    \centering
    \includegraphics[scale=\graphicsscale]{resources/evaluation-charts-\i}
    \caption{}
    \label{fig:evaluation-charts-\i}
\end{subfigure}
}

% caption
\newcommand{\captionvalue}{Bewertung der Aussagen bzgl. der Evaluation des Konzepts}
\caption[\captionvalue]{
\captionvalue
\\\hspace{\textwidth}
\subref{fig:evaluation-charts-a} Ich habe meiner Meinung nach alle Aufgaben erfolgreich erfüllt.
\\\hspace{\textwidth}
\subref{fig:evaluation-charts-b} Die Erfüllung der Aufgaben war anstrengend.
\\\hspace{\textwidth}
\subref{fig:evaluation-charts-c} Ich denke, dass ich die Aufgaben auch ohne vorherige Vorstellung der möglichen Aktionen des Prototyps problemlos lösen könnte.
\\\hspace{\textwidth}
\subref{fig:evaluation-charts-d} Ich hatte das Programm die ganze Zeit unter Kontrolle.
\\\hspace{\textwidth}
\subref{fig:evaluation-charts-e} Die auf \enquote{Drag and Drop} basierende Verschiebungsaktion war verständlich.
\\\hspace{\textwidth}
\subref{fig:evaluation-charts-f} Durch die Einschränkung der freien Positionierung wurde der Aufwand an der Erstellung des Layouts reduziert.
\\\hspace{\textwidth}
\subref{fig:evaluation-charts-g} Die Einschränkung der freien Positionierung fand ich irritierend.
\\\hspace{\textwidth}
\subref{fig:evaluation-charts-h} Für mich war jederzeit erkennbar, an welche Stelle ich die angeklickte Klasse verschieben kann.
\\\hspace{\textwidth}
\subref{fig:evaluation-charts-i} Die automatischen Layout-Updates haben mich in der Erstellung des Diagramms gehindert.
\\\hspace{\textwidth}
\subref{fig:evaluation-charts-j} Das Layout der modellierten Diagramme fand ich ästhetisch ansprechend.
\\\hspace{\textwidth}
\subref{fig:evaluation-charts-k} Ich konnte Fehler, die während der Erfüllung der Aufgaben gemacht wurden, korrigieren.
\\\hspace{\textwidth}
\subref{fig:evaluation-charts-l} Ich bin der Meinung, dass ich die Aufgaben in einer Software ohne Layout-Unterstützung schneller erledigen könnte.
}
\label{fig:evaluation-charts}

\end{figure}

Da das umgesetzte Konzept viele neue Interaktionsparadigmen einsetzt, hat die Interaktion einen wesentlichen Teil der Bewertung gebildet. Dadurch wurde ermöglicht, die Erfüllung des Kriteriums der Benutzerfreundlichkeit zu beurteilen (siehe Abschnitt \ref{eval:user-friendly}).

Mehr als die Hälfte der Teilnehmer waren der Meinung, dass sie den Prototyp während der Erfüllung der Aufgaben unter Kontrolle hatten (siehe Abbildung \ref{fig:evaluation-charts-d}). Weiterhin konnte beobachtet werden, dass der in Abschnitt \ref{subsec:temporary-layer-mechanism} vorgestellte Mechanismus der temporären Schicht bzw. vereinfacht die auf \enquote{Drag and Drop} basierende Verschiebungsaktion von den Teilnehmern als intuitiv angesehen wurde. Dies wurde auch durch eine überwiegende Zustimmung im Fragebogen bestätigt (siehe Abbildung \ref{fig:evaluation-charts-e}).

Da der umgesetzte Ansatz keine freie Positionierung bietet, die von Werkzeugen die ausschließlich das manuelle Layout unterstützen (siehe Abschnitt \ref{sec:manual-layout}) bekannt ist, war die Einschränkung der freien Positionierung der Gegenstand der weiteren Bewertung. Die Mehrheit der Teilnehmer hat angegeben, dass die Einschränkung der freien Positionierung den Aufwand an der Erstellung des Layouts reduziert (siehe Abbildung \ref{fig:evaluation-charts-f}). Bis auf eine Ausnahme wurden die Teilnehmer durch diesen ungewöhnlicher Aspekt der Interaktion nicht verwirrt (siehe Abbildung \ref{fig:evaluation-charts-g}). Diese Resultate beweisen, dass die Einschränkung der freien Positionierung kein Problem dargestellt hat.

Infolge des Einsatzes des baumbasierten Layouts hatte die Verschiebungsaktion zwei Funktionen. Einerseits konnte eine gesamte Vererbungshierarchie durch das Ziehen der Wurzelklasse verschoben werden. Anderseits konnte die Reihenfolge der Geschwisterklassen mit Hilfe der Verschiebungsaktion variiert werden. Obwohl die möglichen Zielpositionen der Verschiebung zum großen Teil für die Teilnehmer erkennbar waren (siehe Abbildung \ref{fig:evaluation-charts-h}), sind die Ergebnisse nicht überzeugend. Deshalb sollte insbesondere in einer allgemeinen Version des Algorithmus über eine potentielle Visualisierung der Verschiebungsoptionen nachgedacht werden.

Die automatischen Layout-Updates haben keine Hinderung in der Erstellung des Diagramms dargestellt (siehe Abbildung \ref{fig:evaluation-charts-i}). Dagegen wurde die Ästhetik des Layouts größtenteils als ansprechend empfunden (siehe Abbildung \ref{fig:evaluation-charts-j}). Die möglichen Ursachen für eine nicht überzeugende Zustimmung sind auf die Überlappung der Pfeilspitzen bei der Vererbungsrelationen und die Umbrüche der Klassennamen zurückzuführen.

Die Antworten bzgl. der Möglichkeit der Fehlerkorrektur während der Erfüllung der Aufgaben unterscheiden sich erheblich (siehe Abbildung \ref{fig:evaluation-charts-k}). Der Grund dafür war eine unterschiedliche Fehlerquote der einzelnen Teilnehmer. Wenn ein Fehler gemacht wurde, der mit Hilfe der im Prototyp verfügbaren Aktionen nicht korrigiert werden konnte, musste die Aufgabe wiederholt werden. Die fehlenden Aktionen werden im Folgenden diskutiert. Trotz der Unvollständigkeit des Prototyps waren die Teilnehmer vorwiegend der Meinung, dass die Layout-Unterstützung ein Zeitersparnis im Vergleich zu anderen Werkzeugen bringt (siehe Abbildung \ref{fig:evaluation-charts-l}). Dadurch wurde der geschaffene Mehrwert des Ansatzes gezeigt.

Wie bereits in Kapitel \ref{chapter:prototype} erwähnt wurde, ist der umgesetzte Prototyp nicht ausgereift und unterstützt viele Funktionen nicht. Die fehlenden Funktionen bilden den Gegenstand für einen weiteren Teil der Nutzerstudie. Einerseits wurde beobachtet, welche nicht implementierten Funktionen die Teilnehmer auszuführen versucht haben. Anderseits wurden die Teilnehmer explizit nach den vermissten Funktionen gefragt. Die Ergebnisse sind in Abbildung \ref{fig:missed-prototype-functions} visualisiert.

\begin{figure}[hbt]
    \centering
    \includegraphics[width=\textwidth]{resources/missed-prototype-functions}
    \caption{Übersicht der vermissten Funktionen im Prototyp}
    \label{fig:missed-prototype-functions}
\end{figure}

Als eine der meist vermissten Funktionen gilt die Möglichkeit des Hinzufügens einer neuen Klasse mit einem Klick bzw. Doppelklick auf das Icon in der Sidebar. Die Einschränkung auf das \enquote{Drag and Drop} wurde als ungenügend eingeschätzt, insbesondere wenn mehrere Klassen auf einmal hinzugefügt werden sollten. Die Behebung dieser Schwachstelle besteht darin, beide Techniken zu unterstützen und bei dem Klick, die neue Klasse an eine vorgegebene Position im Diagramm zu platzieren.

Um die Korrektur von Fehlern zu ermöglichen, müssten Funktionen wie Löschen von Objekten, Manipulation der Vererbungsrelationen und \enquote{Undo} zur Verfügung stehen. Für die objektbezogenen Objekte ist weiterhin die Funktion deren Markierung im Canvas notwendig, die auch für eine gleichzeitige Verschiebung von mehreren Klassen eingesetzt wurde. Die genannten Funktionen bilden eine Gruppe von Standardfunktionen, die in der Regel in einer kompletten Anwendung unterstützt werden sollten und durch die Teilnehmer der Nutzerstudie vermisst wurden. Deren Implementierung im Prototyp für Zwecke dieser Arbeit wurde aus zeitlichen Gründen weggelassen. Dennoch werden alle genannten Funktionen im Konzept berücksichtigt, denn sie können mit Hilfe von Layout-Ereignissen beschrieben werden (siehe Abschnitte \ref{sec:layout-calculation} und \ref{subsubsec:component-events}).

Die Abbildung \ref{fig:missed-prototype-functions} zeigt weiterhin zwei kleinere Probleme der Benutzeroberfläche im Prototyp. In zwei Fällen fänden die Teilnehmer intuitiver, wenn neu hinzugefügten Klassen den Fokus behalten würden, um den Klassennamen auch ohne einen zusätzlichen Doppelklick eingeben zu können. Des Weiteren wurde in einem Fall angemerkt, dass das Werkzeug für die Vererbungsrelation dauerhaft ausgewählt werden könnte. Aufgrund der Kompliziertheit mit einer ständigen Auswahl des Icons in der Sidebar hat die Mehrheit der Teilnehmer zu der Alternative mit der Taste \texttt{CTRL} (siehe Abschnitt \ref{subsec:supported-actions}) gewechselt. Beide Probleme könnten durch einfache Anpassungen im Prototyp behoben werden und haben keinen Einfluss auf das gesamte Konzept. Das Erstellen von Relationen könnte zusätzlich noch mit an die Klassen gebundenen Buttons wie in \textit{Visual Paradigm} durchgeführt werden \cite{14Visual}. Insbesondere würde dies ein angemerktes Problem mit der nicht intuitiven Positionierung von neu hinzugefügten Klassen lösen.

Der Ablauf der Modellierung hat sich je nach dem Teilnehmer und der konkreten Aufgabe unterschieden. Im Wesentlichen wurden die Klassen nacheinander hinzugefügt und jede Klasse direkt umbenannt und in eine Vererbungshierarchie eingeordnet. In einigen Fällen wurde versucht, die Klassen frei zu positionieren. Die Funktionsweise wurde den Teilnehmern vor allem durch das Platzhalter-Objekt sichtbar gemacht und sie haben sich an die Einschränkung der freien Positionierung und das halbautomatische Layout angewöhnt. Ferner wurde die Zentrierung des Inhalts bei der Vergrößerung des Fensters als natürlich empfunden.

Die Nutzerstudie hat bestätigt, dass der entwickelte Ansatz den Prozess der Layout-Erstellung deutlich vereinfacht. Obwohl der umgesetzte Prototyp sehr eingeschränkt war und nur die Modellierung von einfachen Aufgaben ermöglicht hat, haben sich die Möglichkeiten der Interaktion dank der eingesetzten Bedienungskonzepten als intuitiv aufgewiesen. Weiterhin wurden durch die Nutzerstudie Stellen entdeckt, die verbessert bzw. erweitern werden könnten. Die gewonnenen Informationen sind für eine weitere Entwicklung des Prototyps und vor allem des zu Grunde liegenden Ansatzes sehr hilfreich.

\section{Zusammenfassung}
