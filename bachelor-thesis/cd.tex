%%%%%%%%%%%%%%
% CD Content %
%%%%%%%%%%%%%%

\chapter{Inhalt der CD}
\label{chapter:cd-content}

Die auf der beigelegten CD enthaltenen Dateien werden im Folgenden mit ihren Pfaden und kurzen Beschreibungen aufgelistet.

\newenvironment{fileslist}{
    \renewcommand\descriptionlabel[1]{\texttt{##1}}
    \setlength{\leftmargini}{0em}
    \begin{description}[style=nextline]
}{
    \end{description}
}

\section{Schriftliche Arbeit}

\begin{fileslist}

\item[Bachelor-Thesis/bachelor-thesis.pdf] 
Originalversion der Bachelor-Arbeit

\item[Bachelor-Thesis/bachelor-thesis-print.pdf] 
Druckversion der Bachelor-Arbeit

\end{fileslist}

\section{Prototyp}

\begin{fileslist}

\item[Prototype/InteractiveDiagramLayout.app]
Kompilierte Version des Prototyps in Version 0.1.2

\item[Prototype/InteractiveDiagramLayout/] 
Xcode-Projekt und Quellcode des Prototyps in Version 0.1.2

\item[Prototype/Videos/] 
Videos zur Vorführung des implementierten Prototyps

\end{fileslist}

\section{Nutzerstudie}
\label{sec:files-user-study}

\begin{fileslist}

\item[User-Study/task-description.pdf] 
Aufgabenstellung für die Nutzerstudie

\item[User-Study/questionnaire.pdf]
Fragebogen für die Nutzerstudie

\item[User-Study/Completed-Questionnaires/]
Eingescannte ausgefüllte Fragebögen

\item[User-Study/Screen-Recordings/]
Bildschirmaufnahmen der einzelnen Sitzungen

\item[User-Study/evaluation.numbers]
Auswertung der Ergebnisse der Nutzerstudie als Numbers Dokument

\item[User-Study/evaluation.pdf]
Auswertung der Ergebnisse der Nutzerstudie als PDF

\end{fileslist}
