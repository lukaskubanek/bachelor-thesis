%%%%%%%%%%%%%%%%%%%%%%%
% Existing Approaches %
%%%%%%%%%%%%%%%%%%%%%%%

\chapter{Bestehende Ansätze für das Layout von Diagrammen}
\label{chapter:existing-approaches}

Dieses Kapitel widmet sich der Vorstellung von bestehenden Ansätzen für das Layout von Diagrammen. In Abschnitt \ref{sec:categorization} werden die Ansätze zunächst kategorisiert. Anschließend werden in Abschnitten \ref{sec:manual-layout}, \ref{sec:automatic-layout} und \ref{sec:interactive-semi-automatic-layout} die einzelnen Hauptkategorien anhand von Beispielen ausführlich beschrieben und deren Eigenschaften erläutert.

\section{Kategorisierung}
\label{sec:categorization}

Diese Arbeit setzt sich insbesondere mit der Interaktivität der Ansätze für das Layout von Diagrammen auseinander. Aus diesem Grund werden die bestehenden Ansätze nach dem \textbf{Grad der Interaktivität} und der \textbf{Art der Layout-Unterstützung} in drei Hauptkategorien unterteilt.

Zum einen gibt es das interaktive \textbf{manuelle Layout}, das auf der Freihand-Bearbeitung basiert und dem Nutzer ermöglicht, die Layout-Eigenschaften der Objekte im Diagramm wunschgemäß zu verändern, wobei dem Nutzer zusätzlich Layout-Vorschläge präsentiert werden. Das manuelle Layout wird in den meisten grafischen Editoren eingesetzt und wird näher in Abschnitt \ref{sec:manual-layout} erläutert.

Weiterhin gibt es Ansätze für das \textbf{vollautomatische Layout}, die optimale Layouts unter Berücksichtigung von ästhetischen Prinzipien produzieren. Sie haben einen statischen Charakter und können von dem Nutzer nur minimal beeinflusst werden. Somit sind sie nicht für interaktive Umgebungen geeignet. Mit diesen Ansätzen beschäftigt sich Abschnitt \ref{sec:automatic-layout}.

Schließlich gibt es Ansätze für das \textbf{halbautomatische Layout}, die das Layout automatisiert berechnen, lassen sich aber durch den Nutzer interaktiv beeinflussen. Demzufolge kombinieren sie die Vorteile der vorherigen beiden Kategorien. Diese Ansätze werden näher in Abschnitt \ref{sec:interactive-semi-automatic-layout} beschrieben.

\section{Manuelles Layout}
\label{sec:manual-layout}

In Anwendungen zur Erstellung von Diagrammen wie z.B. Vektorgrafik-Software, Präsentationsprogrammen oder CASE-Tools wird ausschließlich das manuelle Layout unterstützt, d.h. die einzelnen Bestandteile können im Diagramm frei positioniert werden. Somit kann der Nutzer ein gewünschtes Layout erreichen, muss dafür aber viele manuelle Schritte tätigen \cite{Eichelberger05Aesthetics}. Da sich der Inhalt des Diagramms während der Erstellung ständig ändert, ist es notwendig, auch das Layout manuell anzupassen. Dies kann zur Frustration des Nutzers führen. Um einige Hürden des manuellen Layouts zu überwinden, bieten die Anwendungen eine Reihe an Funktionen, die dem Nutzer diese Tätigkeit bequemer machen.

Da das manuelle Layout auf direkter Interaktion des Nutzers mit dem Diagramm basiert, haben auch die Hilfsfunktionen einen interaktiven Charakter. Außerdem weisen sie ein temporäres Verhalten auf, d.h. bei ihrer Nutzung werden keinerlei Regeln oder Constraints erstellt, wie das meistens bei den halbautomatischen Ansätzen der Fall ist (siehe Abschnitt \ref{sec:interactive-semi-automatic-layout}).

In Abschnitt \ref{subsec:applications-for-manual-layout} werden zunächst Anwendungen vorgestellt, die eine Grundlage für die Beschreibung der Hilfsfunktionen darstellen. Im Detail werden die Hilfsfunktionen in Abschnitt \ref{subsec:help-functions-for-manual-layout} behandelt. Schließlich werden die Eigenschaften der Ansätze für das manuelle Layout in Abschnitt \ref{subsec:summary-manual-layout} zusammengefasst.

\subsection{Anwendungen als Grundlage}
\label{subsec:applications-for-manual-layout}

Im Folgenden werden drei Anwendungen vorgestellt, an denen die Hilfsfunktionen für das manuelle Layout gezeigt werden.

\subsubsection{OmniGraffle}
\label{subsubsec:omnigraffle}

Die von \textit{OmniGroup}\footnote{\url{http://omnigroup.com}} entwickelte kommerzielle Anwendung \textit{OmniGraffle} ist eines der einfachsten Tools zur Erstellung von Diagrammen unter \textit{Mac OS X} \cite{Olsen10OmniGraffle}. \textit{OmniGraffle} ist sehr flexibel und lässt sich für viele verschiedene Aufgaben einsetzen, z.B. von einem grafischen Entwurf einer Webseite bis zum Zeichnen von Klassendiagrammen. Das liegt vor allem daran, dass für die gezeichneten Diagramme kein semantisches Modell vorliegt und dass die Diagrammbestandteile als reine vektorgrafische Objekte repräsentiert werden. Detaillierte Informationen zu \textit{OmniGraffle} können in \cite{08OmniGraffle}, \cite{Olsen10OmniGraffle} oder auf der offiziellen Webseite\footnote{\url{http://omnigroup.com/omnigraffle}} nachgelesen werden.

Die meisten in Abschnitt \ref{subsec:help-functions-for-manual-layout} beschriebenen Hilfsfunktionen für das manuelle Layout werden anhand von \textit{OmniGraffle} 5\footnote{Die aktuelle Version von \textit{OmniGraffle} ist 6 (\url{http://www.omnigroup.com/blog/omnigraffle-6-is-here}), die Erkenntnisse für diese Arbeit wurden allerdings mit der Version 5 gewonnen.} gezeigt. \textit{OmniGraffle} bietet auch eine Unterstützung für automatisches Layout, die näher in Abschnitt \ref{subsubsec:omnigraffle-auto-layout} beschrieben wird.

\subsubsection{Keynote}
\label{subsubsec:keynote}

\textit{Apple} \textit{Keynote} ist eine Anwendung zur Erstellung von Präsentationen für \textit{Mac OS X} und \textit{iOS}. Es bietet ähnliche Funktionen wie \textit{Microsoft PowerPoint}\footnote{\url{http://office.microsoft.com/en-us/powerpoint/}} und bildet somit eine Alternative zu dem bekannten Präsentationsprogramm. Neben den Achsendiagrammen zur Visualisierung von Werten wie Linien- oder Balkendiagramme ermöglicht es \textit{Keynote} mithilfe von vektorgrafischen Objekten einfache graphbasierte Diagramme zu erstellen. Weitere Informationen sind in \cite{11Keynote} oder auf der offiziellen Webseite\footnote{\url{http://www.apple.com/de/mac/keynote/}} zu finden. 

\textit{Keynote} besitzt ähnliche oder sogar identische Hilfsfunktionen für das manuelle Layout wie \textit{OmniGraffle}. In Abschnitt \ref{subsec:help-functions-for-manual-layout} werden einige Hilfsfunktionen anhand von \textit{Keynote} 6 erklärt. Das automatische Layout wird in \textit{Keynote} nicht unterstützt.

\subsubsection{Visual Paradigm}
\label{subsubsec:visual-paradigm}

\textit{Visual Paradigm} ist ein plattformunabhängiges CASE-Tool von der gleichnamigen Firma und unterstützt neben UML-Diagrammen auch weitere visuelle Sprachen \cite{14Visual}. Das Tool basiert auf einer Freihandbearbeitung und bietet neben Hilfsfunktionen für das manuelle Layout (siehe Abschnitt \ref{subsec:help-functions-for-manual-layout}) auch Unterstützung für automatische Layout-Algorithmen (siehe Abschnitt \ref{subsec:visual-approaches}) \cite{Fuhrmann11On-the-Pragmatics}. Mit einer näheren Beschreibung von \textit{Visual Paradigm} beschäftigen sich \cite{14Visual}, \cite[S.313-314]{Fuhrmann11On-the-Pragmatics} und die offizielle Webseite\footnote{\url{http://www.visual-paradigm.com}}.

\subsection{Hilfsfunktionen für manuelles Layout}
\label{subsec:help-functions-for-manual-layout}

\subsubsection{Raster}
\label{subsubsec:grid}

Viele (unter anderem alle drei in Abschnitt \ref{subsec:applications-for-manual-layout} erwähnten) Visualisierungsprogramme unterstützen die Aufteilung des Diagramms in ein gleichmäßiges Raster, das durch Rasterlinien dargestellt werden kann. Diese Linien können zum Ausrichten von Objekten im Diagramm verwendet werden, das vor allem dann nützlich ist, wenn die Größen der Objekte eine wichtige Rolle spielen wie z.B. in Grundrissen \cite{08OmniGraffle, Olsen10OmniGraffle, 11Keynote, 14Visual}.

\paragraph{Snap-to-Grid}

Die reguläre Verschiebungsaktion von Objekten im Diagramm funktioniert in der Regel wie folgt: Der Nutzer wählt mit dem Mauszeiger ein Objekt aus und schiebt das Objekt mit gedrückter Maustaste auf seine neue Position. Wenn die Funktion \enquote{Snap-to-Grid} nicht aktiviert ist, kann die Zielposition beliebig sein. Während der Verschiebung wird die Position des Objekts kontinuierlich angepasst, sodass der Nutzer ein visuelles Feedback bekommt. Dieser Vorgang ist in Abbildung \ref{fig:omnigraffle-snap-to-grid-off} dargestellt.

\begin{figure}[hbt]
    \newcommand{\subfigurewidth}{0.33\textwidth}
    \newcommand{\graphicswidth}{0.95\linewidth}
    \begin{subfigure}{\subfigurewidth}
        \centering
        \includegraphics[width=\graphicswidth]{assets/omnigraffle-snap-to-grid-off-a}
        \caption{}
    \end{subfigure}
    \begin{subfigure}{\subfigurewidth}
        \centering
        \includegraphics[width=\graphicswidth]{assets/omnigraffle-snap-to-grid-off-b}
        \caption{}
    \end{subfigure}
    \begin{subfigure}{\subfigurewidth}
        \centering
        \includegraphics[width=\graphicswidth]{assets/omnigraffle-snap-to-grid-off-c}
        \caption{}
    \end{subfigure}
    \caption{Verschiebungsaktion mit nicht aktivierter Funktion \enquote{Snap-to-Grid} in \textit{OmniGraffle}}
    \label{fig:omnigraffle-snap-to-grid-off}
\end{figure}

Wenn die Funktion \enquote{Snap-to-Grid} nun aktiviert ist, wird im Unterschied zu dem vorherigen Vorgang die Zielposition so eingeschränkt, dass das verschobene Objekt an den Rasterlinien ausgerichtet wird. Der Mauszeiger kann dabei frei positioniert werden, das Objekt nimmt aber immer die nächste ausgerichtete Position an. Dieser Vorgang ist in Abbildung \ref{fig:omnigraffle-snap-to-grid-on} dargestellt.

\begin{figure}[hbt]
    \newcommand{\subfigurewidth}{0.33\textwidth}
    \newcommand{\graphicswidth}{0.95\linewidth}
    \begin{subfigure}{\subfigurewidth}
        \centering
        \includegraphics[width=\graphicswidth]{assets/omnigraffle-snap-to-grid-on-a}
        \caption{}
    \end{subfigure}
    \begin{subfigure}{\subfigurewidth}
        \centering
        \includegraphics[width=\graphicswidth]{assets/omnigraffle-snap-to-grid-on-b}
        \caption{}
    \end{subfigure}
    \begin{subfigure}{\subfigurewidth}
        \centering
        \includegraphics[width=\graphicswidth]{assets/omnigraffle-snap-to-grid-on-c}
        \caption{}
    \end{subfigure}
    \caption{Verschiebungsaktion mit aktivierter Funktion \enquote{Snap-to-Grid} in \textit{OmniGraffle}}
    \label{fig:omnigraffle-snap-to-grid-on}
\end{figure}

Neben der Verschiebungsaktion wird auch die Aktion der Größenänderung eingeschränkt, indem das manipulierte Objekt nur eine solche Größe annehmen kann, die die Ausrichtung an die Rasterlinien nicht verletzt.

\paragraph{Ausrichten im Raster}

Das Ausrichten an Rasterlinien kann auch nachträglich bewirkt werden. Dies passiert durch die Auswahl des auszurichtenden Objekts und dem Ausführen der Funktion \textit{Ausrichten im Raster}. Die Position und Größe des Objekts werden so angepasst, dass das Objekt an den Rasterlinien ausgerichtet wird.

\subsubsection{Ausrichten und Verteilen von Objekten in Relation zueinander}
\label{subsubsec:alignment-and-distribution}

Ähnlich wie das Raster (Abschnitt \ref{subsubsec:grid}) werden die Funktionen zum Ausrichten und Verteilen von ausgewählten Objekten häufig in Visualisierungsprogrammen eingesetzt. Sie funktionieren wie folgt: Der Nutzer wählt im Diagramm Objekte aus, auf die die Funktion angewendet werden soll. Nachher wählt er eine konkrete Form der Funktion aus, die anschließend auf die ausgewählten Objekt angewendet wird \cite{11Keynote}.

\paragraph{Ausrichten}

Die ausgewählten Objekte können in Relation zueinander ausgerichtet werden. Dafür muss nach der Auswahl von mindestens zwei Objekten die Form der Ausrichtung angegeben werden. Die Objekte können entweder vertikal (linke Kanten, Mittelpunkte oder rechte Kanten) oder horizontal (obere Kanten, Mittelpunkte oder untere Kanten) ausgerichtet werden. Nach dem Ausführen dieser Funktion werden die Positionen der ausgewählten Objekte so angepasst, dass die gewünschte Ausrichtung erfüllt ist \cite{11Keynote, 08OmniGraffle}. Ein Beispiel ist in Abbildung \ref{fig:keynote-horizontal-alignment} zu finden.

\begin{figure}[hbt]
    \newcommand{\subfigurewidth}{0.5\textwidth}
    \begin{subfigure}{\subfigurewidth}
        \centering
        \includegraphics{assets/keynote-horizontal-alignment-a}
        \caption{}
    \end{subfigure}
    \begin{subfigure}{\subfigurewidth}
        \centering
        \includegraphics{assets/keynote-horizontal-alignment-b}
        \caption{}
    \end{subfigure}
    \caption{Anwendung der horizontalen zentrierten Ausrichtung in \textit{Keynote}}
    \label{fig:keynote-horizontal-alignment}
\end{figure}

\paragraph{Verteilen}

Wenn mindestens drei Objekte ausgewählt werden, kann die Funktion zum gleichmäßigen Verteilen der Objekte angewendet werden. Dabei wird die Position der äußeren Objekte fixiert und die Position von umschlossenen Objekten wird so angepasst, dass die vertikalen bzw. horizontalen Abstände zwischen den Objekten gleich sind (siehe Abbildung \ref{fig:keynote-horizontal-distribution}) \cite{11Keynote}.

\begin{figure}[hbt]
    \newcommand{\subfigurewidth}{0.5\textwidth}
    \begin{subfigure}{\subfigurewidth}
        \centering
        \includegraphics{assets/keynote-horizontal-distribution-a}
        \caption{}
    \end{subfigure}
    \begin{subfigure}{\subfigurewidth}
        \centering
        \includegraphics{assets/keynote-horizontal-distribution-b}
        \caption{}
    \end{subfigure}
    \caption{Anwendung der horizontalen Verteilung in \textit{Keynote}}
    \label{fig:keynote-horizontal-distribution}
\end{figure}

\subsubsection{Smart Guides}
\label{subsubsec:smart-guides}

Eine sehr hilfreiche Funktion, die in den meisten Grafik- und Präsentationsprogrammen unterstützt wird, sind Hilfslinien für Ausrichtung und relative Positionierung (engl.: \enquote{Smart Guides}) \cite{11Keynote}. Wenn diese Funktion aktiviert ist, werden während der Manipulation des ausgewählten Objekts farbige Linien im Canvas eingeblendet. Diese Linien geben dem Nutzer ein visuelles Feedback zur geeigneten Ausrichtung des manipulierten Objekts relativ zu anderen Objekten im Diagramm. Außerdem wird ähnlich wie bei der Funktion \textit{Snap-to-Grid} die freie Manipulation eingeschränkt, was in der Regel zu besseren Layout-Ergebnissen führt. Die Hilfslinien werden nach der Beendigung der Aktion wieder ausgeblendet und die Position bzw. Größe des manipulierten Objekts wird angepasst. Genauso wie auch bei allen anderen Hilfsfunktionen für das manuelle Layout handelt es sich bei den Hilfslinien um eine temporäre Layout-Funktion. Somit wird die Ausrichtung der Objekte nur in den Eigenschaften der Objekte (Position und Größe) ausgedrückt und es wird keine explizite Regel erstellt.

\paragraph{Hilfslinien zur Ausrichtung (engl.: \enquote{Alignment Guides})}

Die erste Art der Hilfslinien ist für die Ausrichtung des verschobenen Objekts an den Kanten bzw. Mittelachsen von anderen Objekten in Diagramm nützlich. Die Linien werden während der Verschiebungsaktion eingeblendet, wenn das verschobene Objekt an ein anderes Objekt ausgerichtet wird \cite{11Keynote}.

\begin{figure}[hbt]
    \newcommand{\subfigurewidth}{0.5\textwidth}
    \begin{subfigure}{\subfigurewidth}
        \centering
        \includegraphics{assets/omnigraffle-smart-guides-a}
        \caption{}
        \label{fig:omnigraffle-smart-guides-a}
    \end{subfigure}
    \begin{subfigure}{\subfigurewidth}
        \centering
        \includegraphics{assets/omnigraffle-smart-guides-b}
        \caption{}
        \label{fig:omnigraffle-smart-guides-b}
    \end{subfigure}
    \begin{subfigure}{\subfigurewidth}
        \centering
        \includegraphics{assets/omnigraffle-smart-guides-c}
        \caption{}
        \label{fig:omnigraffle-smart-guides-c}
    \end{subfigure}
    \begin{subfigure}{\subfigurewidth}
        \centering
        \includegraphics{assets/omnigraffle-smart-guides-d}
        \caption{}
        \label{fig:omnigraffle-smart-guides-d}
    \end{subfigure}
    \caption{Hilfslinien zur Ausrichtung während der Verschiebungsaktion in \textit{OmniGraffle}}
    \label{fig:omnigraffle-smart-guides}
\end{figure}

In Abbildung \ref{fig:omnigraffle-smart-guides} wird ein Beispiel der Hilfslinien zur Ausrichtung der unteren Kanten von zwei Objekten während einer Verschiebungsaktion veranschaulicht. Zunächst wird in \ref{fig:omnigraffle-smart-guides-a} das rechte Objekt ausgewählt und nach unten verschoben. Wenn sich die untere Kante des verschobenen Objekts der unteren Kante des inaktiven Objekts nähert, so wie es in \ref{fig:omnigraffle-smart-guides-b} der Fall ist, wird eine Hilfslinie eingeblendet und das verschobene Objekt springt auf eine ausgerichtete Position hin. In \ref{fig:omnigraffle-smart-guides-c} wird die Verschiebung weitergeführt. Dies kann anhand der Position des Mauszeigers erkannt werden. Das Objekt bleibt so lange ausgerichtet, bis der Abstand der Kanten eine bestimmte Schwelle in \ref{fig:omnigraffle-smart-guides-d} überschreitet. Danach wird die Hilfslinie ausgeblendet und eine freie Positionierung ist wieder möglich.

Aus dem vorgestellten Beispiel folgt, dass sich um die untere Kante des inaktiven Objekts ein Bereich bildet, in dem das verschobene Objekt auf die ausgerichtete Position hinspringt. Dies passiert, wenn sich die Geraden in der Nähe befinden und der Abstand unter einer bestimmten Schwelle liegt. Dieser Bereich wird in Abbildung \ref{fig:omnigraffle-smart-guides-snap-area} visualisiert.

\begin{figure}[hbt]
    \centering
    \includegraphics{assets/omnigraffle-smart-guides-snap-area}
    \caption{Visualisierung des Bereichs für die Ausrichtung an die untere Kante des linken Objekts}
    \label{fig:omnigraffle-smart-guides-snap-area}
\end{figure}

Solche Bereiche werden für alle Kanten und Mittelachsen von nicht ausgewählten Objekten in der horizontalen und vertikalen Richtung gebildet. Wenn der Nutzer ein Objekt verschiebt und eine der Kanten oder die Mittelachse des verschobenen Objekts sich in einem der Bereiche befindet, wird seine Position ausgerichtet, ohne den Mauszeiger zu beeinflussen. Das kann auch gleichzeitig für die horizontale und vertikale Richtung der Fall sein. Somit kann z.B. ein Objekt in einem anderen Objekt zentriert werden (siehe Abbildung \ref{fig:omnigraffle-alignment-guides-centering}).

\begin{figure}[hbt]
    \centering
    \includegraphics{assets/omnigraffle-alignment-guides-centering.png}
    \caption{Ausrichten eines Objekts in einem anderen Objekt mithilfe von einer horizontalen und einer vertikalen Hilfslinie zur Ausrichtung in \textit{OmniGraffle}}
    \label{fig:omnigraffle-alignment-guides-centering}
\end{figure}

Eine Veranschaulichung dieser Funktion bietet in Form einer JavaScript-Anwendung das Open-Source-Projekt \textit{Alignment-Guides}\footnote{Quellcode: \url{https://github.com/mrflix/Alignment-Guides}, Demo-Anwendung: \url{http://mrflix.github.io/Alignment-Guides}}.

\paragraph{Abstandshilfslinien (engl.: \enquote{Distance Guides})}

Die zweite Art der Hilfslinien dient dazu, die Abstände zwischen drei Objekten in einer horizontalen oder vertikalen Reihe gleich zu halten. Während der Verschiebung eines der drei Objekten wird ein Bereich gebildet, in dem das Objekt zu der Position hinspringt, in der die Abstände zwischen den einzelnen Objektpaaren gleich sind. Zusätzlich zu den Hilfslinien wird auch der Abstand eingeblendet \cite{11Keynote, Olsen10OmniGraffle}.

\begin{figure}[hbt]
    \centering
    \includegraphics{assets/omnigraffle-distance-guides.png}
    \caption{Abstandshilfslinien während der Verschiebungsaktion in \textit{OmniGraffle}}
    \label{fig:omnigraffle-distance-guides}
\end{figure}

\paragraph{Größenhilfslinien (engl.: \enquote{Sizing Guides})}

Im Unterschied zu den zwei vorher aufgeführten Arten der Hilfslinien wird die letzte Art durch die Aktion der Größenänderung eines Objekts hervorgerufen. Wenn sich die Größe des manipulierten Objekts der Größe eines anderen Objekts im Diagramm nähert und die Differenz unter einer bestimmten Grenze liegt, wird die Größe des anderen Objekts übernommen. Dies funktioniert getrennt für die Breite und Höhe (siehe Abbildung \ref{fig:omnigraffle-sizing-guides}).

\begin{figure}[hbt]
    \centering
    \includegraphics{assets/omnigraffle-sizing-guides.png}
    \caption{Größenhilfslinien während der Aktion der Größenänderung in \textit{OmniGraffle}}
    \label{fig:omnigraffle-sizing-guides}
\end{figure}

\subsubsection{Manuelle Hilfslinien}

Weiterhin bieten \textit{OmniGraffle} und \textit{Keynote} die Funktion der Erstellung von manuellen Hilfslinien an \cite{08OmniGraffle, 11Keynote}. Diese Funktion ist sehr ähnlich zu \textit{Smart Guides}, unterscheidet sich aber wie folgt:

\begin{itemize}
    \item Die Hilfslinien werden vom Nutzer manuell platziert - entweder horizontal oder vertikal.
    \item Die Hilfslinien sind global für das gesamte Diagramm und damit nicht relativ zu einem anderen Objekt im Diagramm.
    \item Die Hilfslinien sind permanent sichtbar, auch wenn keine Verschiebungsaktion stattfindet.\footnote{Die Anzeige der manuellen Hilfslinien kann sowohl in \textit{OmniGraffle} als auch in \textit{Keynote} global aus- und eingeschaltet werden.}
\end{itemize}

Die Gemeinsamkeit besteht darin, dass das Ausrichten an die Hilfslinien identisch wie bei den \textit{Smart Guides} funktioniert (siehe Abschnitt \ref{subsubsec:smart-guides}). Ein Bespiel der manuellen Hilfslinie in \textit{OmniGraffle} wird in Abbildung \ref{fig:omnigraffle-manual-guides} dargestellt.

\begin{figure}[hbt]
    \centering
    \includegraphics{assets/omnigraffle-manual-guides.png}
    \caption{Beispiel einer manuellen Hilfslinie in \textit{OmniGraffle}}
    \label{fig:omnigraffle-manual-guides}
\end{figure}

Wie bereits erwähnt, ist das Ausrichten an die manuellen Hilfslinien nicht persistent und eine Verschiebung der Hilfslinie verursacht keine Verschiebung der ausgerichteten Objekte. Dieses Verhalten wird allerdings in den Tools \textit{Microsoft Visio}\footnote{\url{http://visio.microsoft.com}} und \textit{ConceptDraw}\footnote{\url{http://conceptdraw.com}} unterstützt und wird mithilfe von Einweg-Constraints realisiert \cite[S.20]{Maier12A-Pattern-based}.

\subsubsection{Gleiche Größe der Objekte}

Sehr trivial aber dennoch nützlich ist die Funktion der Einstellung von gleichen Größen für ausgewählte Objekte. Nach der Auswahl von zwei oder mehreren Objekten im Diagramm hat der Nutzer die Möglichkeit, die Breite, die Höhe oder beide Dimensionen gleichzeitig auf die Werte des zuerst ausgewählten Objekts zu setzen (siehe Abbildung \ref{fig:omnigraffle-make-same-size}). Diese Funktion wird in \textit{OmniGraffle} unterstützt \cite{Olsen10OmniGraffle}.

\begin{figure}[hbt]
    \newcommand{\subfigurewidth}{0.5\textwidth}
    \begin{subfigure}{\subfigurewidth}
        \centering
        \includegraphics{assets/omnigraffle-make-same-size-a}
        \caption{}
        \label{fig:omnigraffle-make-same-size-a}
    \end{subfigure}
    \begin{subfigure}{\subfigurewidth}
        \centering
        \includegraphics{assets/omnigraffle-make-same-size-b}
        \caption{}
        \label{fig:omnigraffle-make-same-size-b}
    \end{subfigure}
    \caption{Anwendung der Funktion zur Einstellung von gleichen Größen in \textit{OmniGraffle}}
    \label{fig:omnigraffle-make-same-size}
\end{figure}

\subsubsection{Verknüpfungspunkte für Verbindungen}
\label{subsubsec:connection-points}

Die bisher diskutierten Hilfsfunktionen haben sich ausschließlich mit der Positionierung und der Größenbestimmung der Objekte auseinandergesetzt. Für die graphbasierten Diagramme (siehe Abschnitt \ref{subsec:graph-based-diagrams}) sind neben den Knoten auch Kanten von Interesse, die in der Regel durch Verbindungslinien dargestellt werden. Wenn zwei Objekte mit einer Verbindungslinie verbunden werden, bleiben sie auch dann verbunden, wenn sich die Position oder Größe der Objekte ändert \cite{11Keynote}. In vielen Programmen (unter anderem in \textit{Keynote} und \textit{OmniGraffle}) werden die Verknüpfungspunkte der Verbindungslinie standardmäßig an die Stellen positioniert, an denen eine gedachte Strecke, die die Mittelpunkte beider Objekte verbindet, die Kanten der Objekte schneidet (siehe Abbildung \ref{fig:omnigraffle-connection-points}) \cite{08OmniGraffle}.

\begin{figure}[hbt]
    \newcommand{\subfigurewidth}{0.5\textwidth}
    \newcommand{\graphicswidth}{0.8\linewidth}
    \begin{subfigure}{\subfigurewidth}
        \centering
        \includegraphics[width=\graphicswidth]{assets/omnigraffle-connection-points-a}
        \caption{}
        \label{fig:omnigraffle-connection-points-a}
    \end{subfigure}
    \begin{subfigure}{\subfigurewidth}
        \centering
        \includegraphics[width=\graphicswidth]{assets/omnigraffle-connection-points-b}
        \caption{}
        \label{fig:omnigraffle-connection-points-b}
    \end{subfigure}
    \caption{Verknüpfungspunkte einer Verbindung in \textit{OmniGraffle} \subref{fig:omnigraffle-connection-points-a} mit Veranschaulichung der Berechnung \subref{fig:omnigraffle-connection-points-b}}
    \label{fig:omnigraffle-connection-points}
\end{figure}

\textit{OmniGraffle} bietet die Möglichkeit, mithilfe des Magnetwerkzeugs manuell die Verknüpfungspunkte zu definieren. Dies funktioniert auf zwei verschiedene Arten. Entweder kann der Nutzer die Verknüpfungspunkte beliebig in dem Objekt positionieren oder er wählt eine vordefinierte Anordnung aus. Dazu gehört u.a. die gleichmäßige Verteilung einer festen Anzahl an Verknüpfungspunkten entlang allen Kanten oder die Positionierung eines Verknüpfungspunkts an jedem Eckpunkt. Ein Beispiel der Darstellung von Verknüpfungspunkten ist in Abbildung \ref{fig:omnigraffle-magnets-example} zu finden.

\begin{figure}[hbt]
    \centering
    \includegraphics{assets/omnigraffle-magnets-example.png}
    \caption{Beispiel der manuellen Verknüpfungspunkte in \textit{OmniGraffle}: Magnete an den Eckpunkten (links), drei Magnete pro Kante (Mitte), ein manuell positionierter Magnet (rechts)}
    \label{fig:omnigraffle-magnets-example}
\end{figure}

Sobald mindestens ein Magnet dem Objekt hinzugefügt wurde, wird nicht mehr der Verknüpfungspunkt mithilfe des Mittelpunkts wie oben beschrieben berechnet, sondern die Verbindungslinie wird immer von dem nächsten Verknüpfungspunkt angezogen (siehe Abbildung \ref{fig:omnigraffle-magnets-example}). Alternativ kann der Nutzer ein Ende der Verbindungslinie an einen konkreten Verknüpfungspunkt binden. In dem Fall bleibt die Verbindungslinie an den gewählten Verknüpfungspunkt gebunden, auch wenn ein anderer Verknüpfungspunkt näher ist.

\subsection{Zusammenfassung der Eigenschaften}
\label{subsec:summary-manual-layout}

\subsubsection{Vorteile}

\begin{itemize}

\item
\textbf{Unterstützung der interaktiven Bearbeitung}
Das manuelle Layout ermöglicht freie Bearbeitung der Elemente im Diagramm. Insbesondere können die Layout-Eigenschaften des Objekte manuell angepasst werden, wofür zahlreiche Hilfsfunktionen eingesetzt werden können. Dadurch wird die Interaktion des Nutzers mit dem Diagramm in den Vordergrund gestellt.

\item
\textbf{Hohe Flexibilität}
Aufgrund der Möglichkeit der beliebigen Anpassung der Layout-Ei\-gen\-schaften kann das manuelle Layout sehr präzise gesteuert werden. Somit hat der Nutzer einen großen Einfluss auf das resultierende Layout.

\end{itemize}

\subsubsection{Nachteile}

\begin{itemize}

\item
\textbf{Aufwand an der Layout-Erstellung}
Die Aktionen zur Erstellung des Layouts sind temporär, d.h. nach jeder Änderung des Inhalts eines Diagramms muss das erstellte Layout neu angepasst werden, was in der Regel mit viel Aufwand verbunden ist.

\item
\textbf{Fehlende Unterstützung der Diagrammtypen}
Die Tools, die das manuelle Layout einsetzen, lassen sich in zwei Gruppen unterteilen: CASE-Tools, die die grafische Notation der Diagrammtypen unterstützen und Vektorgrafik-Tools, die eine geringe bis keine Unterstützung der grafischen Notation bringen. Die Syntax- und Semantikregeln für die Layout-Erstellung werden jedoch weder in den CASE-Tools noch in den Vektorgrafik-Tools berücksichtigt und können durch den Nutzer verletzt werden.

\end{itemize}

\section{Automatisches Layout}
\label{sec:automatic-layout}

Den graphbasierten Softwarediagrammen unterliegt die abstrakte Struktur des Graphen (siehe Abschnitt \ref{subsec:graph-based-diagrams}). Mit der grafischen Darstellung von Graphen beschäftigt sich die mathematische Disziplin des Graphzeichnens. Ihre Aufgabe besteht darin, Layout-Algorithmen zu entwerfen, die optimale Layouts in Hinsicht auf ästhetische Prinzipien (siehe Abschnitt \ref{sec:aesthetics-criteria}) erzeugen, d.h. die Layout-Eigenschaften wie Positionen der Knoten und Routen der Kanten berechnen \cite{Eichelberger05Aesthetics, Arvo02Techniques, Siebenhaller03Automatisches, Maier12A-Pattern-based}. Unter dem automatischen Layout für graphbasierte Diagramme sind daher vollautomatische Algorithmen zu verstehen, die für ein gegebenes Diagramm die optimalen Layout-Eigenschaften berechnen und das Diagramm entsprechend anpassen \cite{Fuhrmann11On-the-Pragmatics}.

Um während der Erstellung des Diagramms immer ein optimales Layout zu erhalten, muss der Layout-Algorithmus kontinuierlich nach jeder Änderung des Inhalts entweder automatisch oder manuell aufgerufen werden. Wenn eine Interaktion mit dem resultierenden Diagramm unterstützt wird, hat sie keinerlei Einwirkung auf den Algorithmus und wird bei dem nächsten Aufruf nicht berücksichtigt. In der Regel kann der Nutzer nur bestimmte Parameter des Algorithmus\footnote{z.B. minimale Länge der Kanten im zirkulären Layout-Algorithmus \texttt{circo} aus der Bibliothek \textit{Graphviz} \cite{NorthGansner14Dot-Manual}} anpassen und hat somit nur einen geringen Einfluss auf das Ergebnis des Layout-Prozesses.

Die Ansätze lassen sich in zwei Kategorien nach der Art der Sprache unterteilen, die als Eingabe für den Layout-Algorithmus verwendet wird\footnote{In der Literatur werden Ansätze für das automatische Layout meistens nach den Layout-Algorithmen kategorisiert \cite{Fuhrmann11On-the-Pragmatics, Eichelberger05Aesthetics}. Wie bereits in Abschnitt \ref{sec:categorization} erläutert wurde, richtet sich die Kategorisierung in dieser Arbeit danach, wie die Ansätze zu bedienen sind und welche Art der Interaktion sie unterstützen.}. Zum einen gibt es Ansätze, die aus einer Beschreibung des Diagramms in einer \textbf{textuellen Sprache} unter Anwendung des Layout-Algorithmus eine grafische Repräsentation erzeugen. Zum anderen gibt es Ansätze, die auf Diagramme angewendet werden, die in einer \textbf{visuellen Sprache} modelliert sind. Diese Ansätze verändern die Layout-Eigenschaften der Diagrammelemente direkt.

Mit den textbasierten Ansätzen beschäftigt sich Abschnitt \ref{subsec:text-based-approaches}. Die visuellen Ansätze werden in Abschnitt \ref{subsec:visual-approaches} behandelt. Den speziellen Algorithmen für Klassendiagramme widmet sich Abschnitt \ref{subsec:special-algorithms-for-class-diagrams}. Abschließend werden die Eigenschaften der Ansätze für das automatische Layout in Abschnitt \ref{subsec:summary-automatic-layout} zusammengefasst.

\subsection{Textbasierte Ansätze}
\label{subsec:text-based-approaches}

Die textbasierten Ansätze für das automatische Layout erfordern als Eingabe eine textuelle Beschreibung des Diagramms, die in einer allgemeinen Auszeichnungssprache oder einer domänenspezifischen Sprache formuliert werden kann. Diese Beschreibung wird eingelesen und intern in ein abstraktes Modell umgewandelt, auf das der Layout-Algorithmus angewendet wird. Als Ausgabe wird eine statische Repräsentation des Diagramms in Form eines Bildes geliefert, die jegliche Möglichkeit der Interaktion vermisst. Durch die Entkopplung der Eingabe von der Ausgabe lässt sich das Diagramm nur durch eine Änderung des Quelltexts und einen wiederholten Aufruf des Layout-Algorithmus verändern.

\subsubsection{Graphzeichnen-Tools}
\label{subsubsec:graph-drawing-tools}

Es existiert eine Menge an Bibliotheken, die verschiedene automatische Algorithmen für das Graphzeichnen implementieren, u.a. handelt es sich um hierarchische, kräftebasierte und orthogonale Algorithmen \cite{Maier12A-Pattern-based}. Sie stellen die Funktionalität der automatischen Layout-Be\-rech\-nung für Graphen bereit und können durch andere Programme verwendete werden.  Beispiele für solche Bibliotheken sind \textit{Graphviz}\footnote{\url{http://graphviz.org}}, \textit{yFiles}\footnote{\url{http://www.yworks.com/en/products_yfiles_about.html}} oder \textit{Kieler}\footnote{\url{http://www.informatik.uni-kiel.de/en/rtsys/kieler/}} \cite{Maier12A-Pattern-based}. In dieser Arbeit wird \textit{Graphviz} näher vorgestellt. Insbesondere werden seine Aspekte des textbasierten und visuellen Ansatzes gezeigt.

\textbf{Graphviz} ist ein in der Programmiersprache C geschriebenes Tool für die Visualisierung von Graphen und wurde ursprünglich von \textit{AT\&T} entwickelt. Es besteht aus folgenden Komponenten:

\begin{itemize}
    \item Domänenspezifische \textbf{Sprache Dot}\footnote{Der Aufbau der Sprache wird unter \url{http://www.graphviz.org/content/dot-language} erläutert. Ein Beispiel der Beschreibung eines Graphen ist im Quelltext \ref{lst:graphviz-dot-example-dot} gegeben} für die Beschreibung von Graphen
    \item Ein Satz von \textbf{Layout-Algorithmen}: u.a. \textit{dot}, \textit{neato}, \textit{fdp} oder \textit{circo} \cite{Gansner14Using, NorthGansner14Dot-Manual}
    \item Eine \textbf{Software-Bibliothek}, die die Funktionalität der Layout-Algorithmen und der grafischen Ausgabe bereitstellt und sich in andere Programme einbinden lässt \cite{Gansner14Using}
    \item Ein Satz von \textbf{Kommandozeilen-Tools} für die Anwendung der Layout-Algorithmen und für die grafische Ausgabe \cite{NorthGansner14Dot-Manual}
    \item Eine \textbf{GUI-Anwendung}, die über dieselben Funktionen wie die Kommandozeilen-Tools verfügt
\end{itemize}

Um einen Layout-Algorithmus auf einen Graph mithilfe der Komandozeilen-Tools oder der GUI-Anwendung anwenden zu können, muss der Graph in der Sprache Dot beschrieben werden. Ein Beispiel eines einfachen Graphen mit vier Knoten und drei Kanten ist im Quelltext \ref{lst:graphviz-dot-example-dot} gegeben.

\lstinputlisting[
    caption={Beschreibung eines Graphen in Dot (\texttt{graphviz-dot-example.dot})},
    label={lst:graphviz-dot-example-dot}
]{assets/graphviz-dot-example.dot}

Durch einen Aufruf des Kommandozeilen-Tools \texttt{dot} wird die oben aufgelistete Dot-Quelldatei eingelesen und der beschriebene Graph wird in Form eines internen Modells instanziiert, worauf der Layout-Algorithmus \textit{dot} angewendet wird. Anschließend wird mit einem \enquote{Renderer} die graphische Repräsentation in einer Datei erzeugt, welches in Abbildung \ref{fig:graphviz-dot-example} dargestellt ist \cite{Gansner14Using}.

\begin{figure}[hbt]
    \centering
    \includegraphics[scale=0.75]{assets/graphviz-dot-example.png}
    \caption{Das Resultat des Aufrufs des Kommandozeilen-Tools \texttt{dot}}
    \label{fig:graphviz-dot-example}
\end{figure}

Alternativ kann die Dot-Quelldatei in der GUI-Anwendung geöffnet werden. In dem Fall wird das Ergebnis nicht direkt in eine Datei gespeichert, sondern als PDF in einem Fenster der Anwendung angezeigt.

Das oben diskutierte Beispiel zeigt, dass das resultierende Diagramm nicht interaktiv ist. Eine Änderung des Graphen ist nur in der Quelldatei möglich und ist mit einem wiederholten Aufruf des Kommandozeilen-Tools oder Laden der Datei in der GUI-An\-wen\-dung verbunden. Diese Schwachstelle wird durch 3rd-Party Editoren mit der integrierten grafischen Ausgabe wie etwa \textit{Leonhard}\footnote{\textit{Leonhard} ist ein grafischer Editor für \textit{Graphviz}, der unter älteren Versionen von \textit{Mac OS X} funktioniert. Weitere Information sind unter \url{http://algorithmique.net/leonhard.html} und \url{https://github.com/glejeune/Leonhard} zu finden.} oder \textit{WebGraphviz}\footnote{\url{http://webgraphviz.com}} verbessert. In \textit{Leonhard} wird die Übersetzung nach jeder Änderung des Quelltexts sogar automatisch gestartet.

Trotz der fehlenden Interaktivität des resultierenden Diagramms kann das Layout von dem Nutzer beeinflusst werden, indem der Layout-Algorithmus gewählt wird bzw. seine Parameter in der Quelldatei angepasst werden \cite{NorthGansner14Dot-Manual}. Im oben genannten Beispiel wurde der Layout-Algorithmus \textit{dot} verwendet und die Richtung des Graphen in der Dot-Quelldatei mit dem Befehl \texttt{rankdir = LR} angepasst, sodass der Graph von links nach rechts gezeichnet wird. Die konkreten in \textit{Graphviz} unterstützten Layout-Algorithmen werden in \cite[S.22]{Gansner14Using} beschrieben und in Abschnitt \ref{subsubsec:omnigraffle-auto-layout} zum Teil visualisiert.

\subsubsection{Textbasierte UML-Tools}

Neben den textbasierten Tools für das Graphzeichnen gibt es Tools, die für spezifische Domänen ausgelegt sind. An dieser Stelle werden kurz zwei textbasierte UML-Tools vorgestellt, die aus einer textuellen Beschreibung in einer speziellen Sprache grafische UML-Diagramme erzeugen und intern für die Layout-Berechnung die oben vorgestellte Bibliothek \textit{Graphviz} verwenden:

\begin{itemize}
    \item \textbf{PlantUML}\footnote{\url{http://plantuml.sourceforge.net}} ist eine Java-Bibliothek, die für die Beschreibung von UML-\-Dia\-grammen die gleichnamige Sprache verwendet. Diese Sprache wird näher in \cite{Roques10Drawing} behandelt. Neben den vielen Anwendungen\footnote{Die bekannten Anwendungen sind unter \url{http://plantuml.sourceforge.net/running.html} aufgelistet.} bietet \textit{PlantUML} einen Online-Editor\footnote{\textit{PlantUML} Server: \url{http://www.plantuml.com/plantuml}} an, der es ermöglicht, UML-Diagramme direkt im Browser zu erstellen und als Bilder zu exportieren.
    \item \textbf{yUML}\footnote{\url{http://yuml.me}} ist ein Online-Editor zur Erstellung von UML-Diagrammen im Browser. Im Unterschied zu \textit{PlantUML} verwendet \textit{yUML} eine anschauliche zeichenbasierte Sprache\footnote{Eine Übersicht der Syntax für Klassendiagramme: \url{http://yuml.me/diagram/scruffy/class/samples}}. Die Eingabe erfolgt über ein Textfeld und wird kontinuierlich in ein Bild übersetzt, das jederzeit exportiert werden kann. Alternativ kann eine URL-Adresse generiert \cite{Fuhrmann11On-the-Pragmatics} werden, unter der das gezeichnete Diagramm online verfügbar ist.
\end{itemize}

\subsection{Visuelle Ansätze}
\label{subsec:visual-approaches}

Die visuellen Ansätze für das automatische Layout unterscheiden sich von den textbasierten darin, dass der Layout-Algorithmus auf eine visuelle Sprache angewendet wird und dass das berechnete Layout direkt die Eingabe verändert. Diese Ansätze werden in der Regel in visuellen Editoren eingesetzt, die eine unmittelbare Bearbeitung des Diagramms unterstützen.

Der Ablauf ist wie folgt: Zunächst wird ein Diagramm in einer bestimmten visuellen Sprache (z.B. ein Klassendiagramm in der Sprache UML) modelliert. Danach wird ein Layout-Algorithmus (in der Regel manuell) ausgeführt, der das neue Layout für alle Diagrammbestandteile berechnet. Anschließend wird das modellierte Diagramm dermaßen angepasst, sodass es das berechnete Layout annimmt.

Die Entkoppelung der Ein- und Ausgabe, wodurch sich die textuellen Ansätze auszeichnen (siehe Abschnitt \ref{subsec:text-based-approaches}), entfällt an dieser Stelle und da das resultierende Diagramm interaktiv bleibt, kann es durch den Nutzer weiterhin verändert bzw. erweitert werden. Um nach jeder Änderung des Diagramms ein automatisch berechnetes Layout zu erhalten, muss allerdings der Layout-Algorithmus jedes Mal erneut gestartet werden.

Die Algorithmen für das automatische Layout sind im Allgemeinen nicht ideal und die Nutzer neigen dazu, das berechnete Layout für persönliche Präferenzen anzupassen, um ein mentales Modell (siehe Abschnitt \ref{subsec:mental-map}) bzw. eine sekundäre Notation (siehe Abschnitt \ref{subsec:layout}) zu verwalten. Alle nachträglich manuell getätigten Layout-Änderungen werden bei einem erneuten Aufruf des Layout-Algorithmus verworfen und somit steht der Nutzer vor der Entscheidung, ob er nach jedem Aufruf des Algorithmus die Anpassungen wiederholt durchführt oder auf das automatische Layout komplett verzichtet \cite[S.119ff]{Eiglsperger04Automatic}.

\subsubsection{Automatisches Layout in OmniGraffle}
\label{subsubsec:omnigraffle-auto-layout}

Wie bereits in Abschnitt \ref{subsubsec:graph-drawing-tools} beschrieben wurde, bietet \textit{Graphviz} eine Software-Bibliothek an, die sich in andere Programme einbinden lässt. Das Visualisierungsprogramm \textit{OmniGraffle} (siehe Abschnitt \ref{subsubsec:omnigraffle}) macht sich dies zunutze und verfügt über eine Funktion für das automatische Layout. Für die Layout-Berechnung verwendet diese die Algorithmen aus \textit{Graphviz}, nämlich den hierarchischen, kraftbasierten, zirkulären und radialen Algorithmus \cite{Olsen10OmniGraffle}. Nachdem die Funktion des automatischen Layouts für ein Diagramm aktiviert wird, kann der Layout-Algorithmus ausgewählt und seine Parameter eingestellt werden\footnote{Die konkreten möglichen Parameter der Algorithmen sind in \cite[S.74]{08OmniGraffle} nachzuschauen.}. In Abbildung \ref{fig:omnigraffle-automatic-layout} wird ein Beispiel der Anwendung von allen verfügbaren Layout-Algorithmen auf ein Diagramm illustriert.

\begin{figure}[hbt]
    \newcommand{\subfigureshortwidth}{0.28\textwidth}
    \newcommand{\subfigurelongwidth}{0.40\textwidth}
    \newcommand{\graphicsscale}{0.7}
    \centering
    \begin{subfigure}{\subfigureshortwidth}
        \centering
        \includegraphics[scale=\graphicsscale]{assets/omnigraffle-automatic-layout-a}
        \caption{Manuell erstelltes Layout}
        \label{fig:omnigraffle-automatic-layout-a}
    \end{subfigure}
    \begin{subfigure}{\subfigureshortwidth}
        \centering
        \includegraphics[scale=\graphicsscale]{assets/omnigraffle-automatic-layout-b}
        \caption{Hierarchisches Layout von oben nach unten}
        \label{fig:omnigraffle-automatic-layout-b}
    \end{subfigure}
    \begin{subfigure}{\subfigureshortwidth}
        \centering
        \includegraphics[scale=\graphicsscale]{assets/omnigraffle-automatic-layout-c}
        \caption{Hierarchisches Layout von links nach rechts mit angepassten \enquote{Ranks} für einzelne Knoten}
        \label{fig:omnigraffle-automatic-layout-c}
    \end{subfigure}
    \begin{subfigure}{\subfigureshortwidth}
        \centering
        \includegraphics[scale=\graphicsscale]{assets/omnigraffle-automatic-layout-d}
        \caption{Kraftbasiertes Layout}
        \label{fig:omnigraffle-automatic-layout-d}
    \end{subfigure}
    \begin{subfigure}{\subfigurelongwidth}
        \centering
        \includegraphics[scale=\graphicsscale]{assets/omnigraffle-automatic-layout-e}
        \caption{Zirkuläres Layout}
        \label{fig:omnigraffle-automatic-layout-e}
    \end{subfigure}
    \begin{subfigure}{\subfigureshortwidth}
        \centering
        \includegraphics[scale=\graphicsscale]{assets/omnigraffle-automatic-layout-f}
        \caption{Radiales Layout}
        \label{fig:omnigraffle-automatic-layout-f}
    \end{subfigure}
    \caption{Beispiele der Anwendung von automatischen Layout-Algorithmen in OmniGraffle}
    \label{fig:omnigraffle-automatic-layout}
\end{figure}

Der eingestellte Layout-Algorithmus kann entweder durch die manuelle Auswahl im Menü, die Veränderung der Parametern des Algorithmus oder das Hinzufügen oder Löschen einer Kante zwischen zwei Knoten erneut aufgerufen werden \cite[S.43]{Wybrow08Using}. Danach wird das Layout berechnet und mithilfe einer Animation auf das Diagramm angewendet, wobei die durchgeführten manuellen Anpassungen verworfen werden. \textit{OmniGraffle} erweitert die Funktion des automatischen Layouts um weitere Tools. So kann z.B. die Struktur des Diagramms parallel in einer Outline bearbeitet oder ein Werkzeug für schnelle Erstellung eines Diagramms verwendet werden \cite{08OmniGraffle}.

\subsubsection{Automatisches Layout in Visual Paradigm}

Ähnlich wie \textit{OmniGraffle} basiert das CASE-Tool \textit{Visual Paradigm} (siehe Abschnitt \ref{subsubsec:visual-paradigm}) auf der freien Positionierung der Elemente im Diagramm und verfügt über die Funktion der automatischen Layout-Berechnung \cite{14Visual}. Dem Nutzer steht eine Vielzahl von Layout-Algorithmen zur Verfügung, deren Eigenschaften wunschgemäß eingestellt werden können \cite{Fuhrmann11On-the-Pragmatics}. Im Unterschied zu der Aktivierung der automatischen Layout-Berechnung für das gesamte Diagramm in \textit{OmniGraffle} (siehe Abschnitt \ref{subsubsec:omnigraffle-auto-layout}) ist die Berechnung in \textit{Visual Paradigm} einmalig und kann auf eine Untermenge des Diagramms angewendet werden. Sie wird manuell durch eine Auswahl im Kontextmenü gestartet, wobei der konkrete Layout-Algorithmus entweder manuell oder automatisch gewählt werden kann. Aus diesem Grund versagt diese Funktion im Bereich der Interaktivität. Eine weitere Schwachstelle dieser Funktion ist die mangelhafte bis fehlende Unterstützung der Semantik der Elemente. Ein Beispiel der Anwendung der automatischen Layout-Berechnung auf ein Klassendiagramm ist in Abbildung \ref{fig:visual-paradigm-auto-layout} dargestellt.

\begin{figure}[hbt]
    \newcommand{\leftsubfigurewidth}{0.4\textwidth}
    \newcommand{\rightsubfigurewidth}{0.59\textwidth}
    \newcommand{\graphicsscale}{0.58}
    \centering
    \begin{subfigure}{\leftsubfigurewidth}
        \centering
        \includegraphics[scale=\graphicsscale]{assets/visual-paradigm-auto-layout-a}
        \caption{Manuelles Layout}
        \label{fig:visual-paradigm-auto-layout-a}
    \end{subfigure}
    \begin{subfigure}{\rightsubfigurewidth}
        \centering
        \includegraphics[scale=\graphicsscale]{assets/visual-paradigm-auto-layout-b}
        \caption{Automatisches Layout}
        \label{fig:visual-paradigm-auto-layout-b}
    \end{subfigure}
    \caption{Ein Beispiel der automatischen Layout-Berechnung für ein einfaches Klassendiagramm in \textit{Visual Paradigm}}
    \label{fig:visual-paradigm-auto-layout}
\end{figure}

\subsection{Spezielle Algorithmen für Klassendiagramme}
\label{subsec:special-algorithms-for-class-diagrams}

Die Berechnung des automatischen Layouts von Klassendiagrammen bildet den Gegenstand für viele Forschungsarbeiten wie z.B. \cite{Eichelberger05Aesthetics}, \cite{Siebenhaller03Automatisches} sowie \cite{Eiglsperger04Automatic}. Die darin vorgestellten Algorithmen haben den gleichen statischen Charakter wie die Algorithmen zum Graphzeichnen, sind allerdings für Klassendiagramme spezialisiert, indem deren Syntax und unter Umständen auch Semantik bei der Layout-Berechnung berücksichtigt werden. Eine grundlegende Rolle spielen für die automatischen Layout-Algorithmen die ästhetischen Prinzipien \cite{Maier12A-Pattern-based}. Da die ästhetischen Prinzipien für Graphen an dieser Stelle nicht ausreichend sind \cite[S.79]{Eichelberger05Aesthetics}, werden sie in den aufgeführten Arbeiten für Klassendiagramme erweitert. Ihre Zusammenfassung wurde in Abschnitt \ref{subsec:aesthetics-criteria-class-diagrams} aufgeführt.

Die genannten Arbeiten machen sich bestehende Algorithmen zum Graphzeichnen zunutze, passen diese für die Klassendiagramme an und setzen sie in der Regel in Form einer Bibliothek um. Zu den verfügbaren algorithmischen Ansätzen gehören Topology-Shape-Metrics \cite[S.33]{Siebenhaller03Automatisches}, hierarchische Algorithmen und kraftbasierte Algorithmen \cite[S.32ff]{Eichelberger05Aesthetics}. Die Grundlage für \cite{Eichelberger05Aesthetics} bildet der hierarchische Sugiyama Algorithmus. Dagegen basieren \cite{Siebenhaller03Automatisches} und \cite{Eiglsperger04Automatic} auf dem Topology-Shape-Metrics Ansatz, der den Prozess der automatischen Layout-Berechnung in mehrere Schritte unterteilt.

Obwohl sich die Algorithmen in den erwähnten Arbeiten intern unterschiedlich verhalten, besteht ihre Funktion in der automatischen Berechnung der Layout-Eigenschaften für die Elemente eines Klassendiagramms, welches in der Regel anhand einer Instanz des UML-Metamodells beschrieben wird. Einerseits lassen sie sich in visuelle Editoren einbauen und können somit den visuellen Ansätzen für das automatische Layout (siehe Abschnitt \ref{subsec:visual-approaches}) zugeordnet werden. Anderseits kann das Modell textuell in Form einer XMI-Datei\footnote{XML Metadata Interchange (\url{http://www.omg.org/spec/XMI})} repräsentiert werden. Daher ist die Einordnung zu den textuellen Ansätzen für das automatische Layout (siehe Abschnitt \ref{subsec:text-based-approaches}) ebenfalls möglich.

Trotz der in \cite{Eiglsperger04Automatic} beschriebenen möglichen Erweiterung des Algorithmus für einen interaktiven Einsatz sind die in allen oben genannten Arbeiten vorgestellten Algorithmen prinzipiell nicht für einen interaktiven Einsatz konzipiert und fokussieren sich eindeutig auf die Umsetzung der ästhetischen Prinzipien. Insbesondere sind sie für Szenarien geeignet, in denen das Klassendiagramm nicht durch den Nutzer erstellt wird, sondern als ein Modell vorliegt, wofür das Layout berechnet werden soll. Beispiele für solche Anwendungsfälle sind Dokumentationswerkzeuge (z.B. \textit{Doxygen}\footnote{\url{http://www.stack.nl/~dimitri/doxygen/manual/diagrams.html}}), Werkzeuge zur Generierung von Diagrammen oder Analyse-Werkzeuge zum \enquote{Reverse Engineering} \cite{Eiglsperger04Automatic}.

\subsection{Zusammenfassung der Eigenschaften}
\label{subsec:summary-automatic-layout}

\subsubsection{Vorteile}

\begin{itemize}

\item
\textbf{Automatische Layout-Berechnung}
Das automatische Layout wird, wie bereits der Name sagt, automatisch berechnet und verringert somit den Aufwand an manueller Layout-Erstellung durch den Nutzer.

\item
\textbf{Einhaltung der ästhetischen Prinzipien}
Die Algorithmen für das automatische Layout von Diagrammen liefern in der Regel optimale Layouts im Bezug auf die ästhetischen Prinzipien \cite{Maier12A-Pattern-based}. Durch die Spezialisierung der Algorithmen kann sogar das Einhalten der syntaktischen Strukturregeln der konkreten Diagrammtypen erreicht werden.

\item
\textbf{Berechnung des initialen Layouts}
Des Weiteren sind die Algorithmen für das automatische Layout von Diagrammen für die Berechnung des initialen Layouts ausgelegt. Dies ist insbesondere in den Anwendungsfällen hilfreich, in denen nur der Inhalt aber kein Layout des Diagramms vorliegen.

\item
\textbf{Möglichkeit der Einbindung in automatisierte Prozesse}
Aufgrund des statischen Charakters, der definierten Eingabesprachen und der Eignung für die Berechnung des initialen Layouts sind die Ansätze für das automatische Layout für die Einbindung in automatisierte Prozesse geeignet.

\end{itemize}

\subsubsection{Nachteile}

\begin{itemize}

\item
\textbf{Fehlende Interaktivität}
Das automatische Layout ist für eine interaktive Bearbeitung der Diagramme nicht konzipiert, weil die direkte Interaktion des Nutzers mit dem Diagramm mangelhaft oder gar nicht unterstützt wird. Obwohl sich die visuellen Ansätze in interaktiven Umgebungen wie etwa visuellen Editoren einsetzen lassen, sind sie dafür nicht geeignet \cite[S.22ff]{Maier12A-Pattern-based} \cite[S.4]{DwyerMarriott08Interactive}. Insbesondere zeichnet sich dies dadurch aus, dass der Prozess der Erstellung eines Diagramms nicht gefördert wird. Die interaktiven Änderungen des Inhalts bzw. des Layouts des Diagramms werden von dem Layout-Algorithmus nicht berücksichtigt und somit kann bei dem Aufruf des Algorithmus das mentale Modell (siehe Abschnitt \ref{subsec:mental-map}) bzw. die sekundäre Notation (siehe Abschnitt \ref{subsec:layout}) zerstört werden \cite{Eiglsperger04Automatic}. Des Weiteren ist es notwendig, den Algorithmus manuell nach jeder Änderung erneut zu starten. Im Unterschied zu den visuellen Ansätzen bieten die textuellen Ansätze keine direkte Form der Interaktion an, was durch die unterschiedlichen Eingabe- und Ausgabesprachen und der damit zusammenhängenden Entkoppelung der Eingabe und Ausgabe bedingt ist.

\item
\textbf{Geringer Einfluss auf das resultierende Layout}
Da das automatische Layout auf statischen Algorithmen basiert, die sich in der Regel nur durch die Anpassung der Parameter beeinflussen lassen, werden die Layout-Präferenzen des Nutzers nicht berücksichtigt. Die mögliche Kontrolle des Ergebnisses des Layout-Prozesses ist somit sehr mangelhaft \cite{GladischSchumann14Semi-Automatic}.

\item
\textbf{Mangelhafte Berücksichtigung der Syntax- und Semantikregeln}
Die syntaktischen und semantischen Layout-Regeln der konkreten Diagrammtypen werden von den allgemeinen Algorithmen für das Layout von graphbasierten Diagrammen nicht berücksichtigt und können dadurch verletzt werden. Diese Schwachstelle wird durch die Anpassung der Algorithmen für konkrete Diagrammtypen beseitigt.

\end{itemize}

\section{Interaktives halbautomatisches Layout}
\label{sec:interactive-semi-automatic-layout}

Wie bereits in diesem Kapitel präsentiert wurde, unterstützen die meisten Editoren zur Erstellung von Diagrammen Hilfsfunktionen für das manuelle Layout (siehe Abschnitt \ref{sec:manual-layout}) und integrieren eventuell zusätzlich auch automatische Layout-Algorithmen (siehe Abschnitt \ref{sec:automatic-layout}). Das manuelle Layout ist sehr intuitiv, zeichnet sich durch die direkte Interaktion des Nutzers mit dem Diagramm aus, vermisst aber eine automatisierte Layout-Berechnung und Berücksichtigung der ästhetischen Prinzipien. Dieses Problem wird durch das automatische Layout angegangen, was aber im Bereich der Interaktivität versagt \cite{GladischSchumann14Semi-Automatic}. Die Vorteile der Ansätze aus beiden genannten Kategorien lassen sich kombinieren und bilden eine neue Kategorie der Ansätze für die halbautomatische Layout-Unterstützung.

Diese Ansätze sind für interaktive Umgebungen ausgelegt, die eine sequenzielle Modifizierung des Diagramms durch den Nutzer unterstützen \cite{Arvo02Techniques, GladischSchumann14Semi-Automatic, Wybrow08Using}. Das Layout wird während der Bearbeitung des Diagramms mithilfe von dynamischen Algorithmen kontinuierlich berechnet und inkrementell angepasst. Neben dem Inhalt kann der Nutzer auch einige Aspekte des Layouts beeinflussen, z.B. durch direkte Positionierung von Knoten bzw. Kanten oder durch eine andere Form von Feedback \cite{Arvo02Techniques}.

\subsection{Struktur-basierte benutzergesteuerte Ansätze}
\label{subsec:structure-based-user-controlled-approaches}

Die erste Kategorie der Ansätze für das interaktive halbautomatische Layout bilden die Struktur-basierten benutzergesteuerten Ansätze, die dem Nutzer ermöglichen, das Layout des Diagramms durch Erstellung und Verwaltung von Strukturregeln\footnote{Ein Beispiel für eine solche Strukturregel ist die in Abschnitt \ref{subsubsec:alignment-and-distribution} beschriebene gleichmäßige Verteilung der ausgewählten Objekte in Relation zueinander.} zu beeinflussen. Diese Strukturregeln erinnern an die Hilfsfunktionen für das manuelle Layout (siehe Abschnitt \ref{subsec:help-functions-for-manual-layout}), sind aber dahingegen persistent \cite{Wybrow08Using}. Sie werden bei der Berechnung des Layouts durch einen dynamischen Layout-Algorithmus berücksichtigt und eingehalten.

Die Struktur-basierten benutzergesteuerten Ansätze lassen sich in zwei Gruppen unterteilen: in Constraint-basierte und Pattern-basierte Ansätze. Beide Gruppen werden im Folgenden vorgestellt.

\subsubsection{Constraint-basierte Ansätze}
\label{subsubsec:constraint-based-approaches}

Die Beschreibung der Strukturregeln kann mithilfe von \textbf{Constraints} erfolgen. Dies macht sich der in \cite{Wybrow08Using} beschriebene Ansatz zunutze und stellt einen Algorithmus für die kontinuierliche Layout-Berechnung in einem interaktiven Editor anhand von Constraints vor. Die Grundidee basiert darauf, dass der Nutzer persistente Constraints für die Beziehungen von ausgewählten Knoten erstellt (z.B. Verteilung oder Ausrichtung der Knoten) und somit deren Layout beschreibt. Die Einhaltung der Constraints wird durch einen \textbf{Constraintlöser}\footnote{Für eine Übersicht der Typen von Constraintlösern ist \cite[S.18ff]{Maier12A-Pattern-based} nachzuschauen.} gewährleistet, der das valide Layout anhand der Zusammensetzung von allen erstellten Constraints berechnet.

Die freie Positionierung der Elemente im Diagramm ist grundsätzlich 
möglich, wird aber durch die erstellten Constraints eingeschränkt. Das zeichnet sich z.B. dadurch aus, dass bei der Bewegung eines Knotens, der mit einem oder mehreren Constraints mit anderen Knoten in Beziehung gesetzt ist, auch die zusammenhängende Knoten automatisch mitbewegt werden, sodass die Constraints eingehalten bleiben. Weiterhin können auch die Parameter der erstellten Constraints durch den Nutzer eingestellt werden. Somit zeichnet sich dieser Ansatz durch eine große Flexibilität aus.

In Abbildung \ref{fig:dunnart-screenshot} ist ein Screenshot des Editors \textit{Dunnart}\footnote{\url{http://dunnart.org}} dargestellt, welcher den Ansatz aus \cite{Wybrow08Using} implementiert. Die rechte Sidebar beinhaltet eine Palette mit Buttons für die Erstellung von Constraints. In der Mitte befindet sich der Canvas, in dem ein hierarchisches Diagramm modelliert ist. Die blauen Linien um das eigentliche Diagramm dienen der Visualisierung von erstellten Constraints und ermöglichen eine Manipulation mit ihnen.

\begin{figure}[hbt]
    \centering
    \includegraphics[width=0.9\textwidth]{assets/dunnart-screenshot}
    \caption{Ein Screenshot des Constraint-basierten Editors \textit{Dunnart} mit dem Beispiel eines hierarchischen Diagramms}
    \label{fig:dunnart-screenshot}
\end{figure}

Die Constraint-basierte Ansätze sind darüber hinaus mit Problemen verbunden. Zum einen kann die Menge der Constraints unvollständig sein oder es werden unzulässige Constraints erstellt. Dies muss in dem Editor entsprechend behandelt werden, indem ein mögliches Layout gewählt wird bzw. die unzulässigen Constraints mit einer Form von Feedback versehen werden. Zum anderen entstehen Probleme mit der Performance, da der Algorithmus und damit auch der Aufruf des Constraintlösers  nach jeder kleinen Änderung ausgeführt wird \cite{Maier12A-Pattern-based}.

\subsubsection{Pattern-basierter Ansatz}
\label{subsubsec:pattern-based-approach}

Der in \cite{Maier12A-Pattern-based} und \cite{MaierMinas10Combination} präsentierte Pattern-basierte Ansatz für das Layout von Diagrammen drückt die Strukturregeln in Form von \textbf{Layout-Patterns} aus. Dieser Ansatz kann für beliebige visuellen Sprachen eingesetzt werden, die mit Metamodellen beschrieben werden können. Dabei besteht das sprachenspezifische Metamodell aus zwei Teilen, nämlich den Metamodellen für die abstrakte und konkrete Syntax. Das zuletzt genannte Metamodell beschreibt die visuellen Eigenschaften der Sprache und wird daher durch die Layout-Berechnung beeinflusst.

Die Beschreibung der Layout-Patterns erfolgt allerdings nicht mithilfe der sprachenspezifischen Metamodellen, sondern mit sprachenunabhängigen \textbf{Pattern-spezifischen Metamodellen}. In \cite{Maier12A-Pattern-based} werden u.a. Pattern-spezifische Metamodelle für Mengen von Elementen, für geordnete Listen von Elementen und für Graphen präsentiert. Zu den darauf aufbauenden Layout-Patterns gehören u.a. horizontale bzw. vertikale Ausrichtung (Menge von Elementen), gleicher horizontaler bzw. vertikaler Abstand (geordnete Liste von Elementen) und die Algorithmen zum Graphzeichnen für das baumbasierte, hierarchische oder zirkuläre Layout (Graph). Eine komplette Liste der vorgestellten Layout-Patterns ist unter \cite[S.55]{Maier12A-Pattern-based} zu finden.

Bei der Instanziierung eines Layout-Patterns wird das sprachenspezifische Modell auf ein Pat\-tern-spezifisches Modell abgebildet und auf gegebene bzw. alle Elemente des Diagramms angewendet\footnote{Dieses Sachverhalt wird mithilfe eines Beispiels in \cite[S.59ff]{Maier12A-Pattern-based} anschaulich gemacht.}. Dadurch wird eine Wiederverwendung der Layout-Patterns für diverse visuellen Sprachen gewährleistet.

Weiterhin enthalten die Layout-Patterns Prädikate, die für ein valides Layout des Diagramms erfüllt sein müssen. Die Erfüllung der Prädikate erfolgt durch die Anwendung von Regeln, die die beteiligten Layout-Variablen anpassen und ebenso von den Layout-Patterns gekapselt werden. Die Regeln können bestehende Algorithmen für das Layout von Diagrammen wiederverwenden, u.a. auch Algorithmen zum Graphzeichnen (siehe Abschnitt \ref{subsubsec:graph-drawing-tools}) und Constraints-basierte Algorithmen (siehe Abschnitt \ref{subsubsec:constraint-based-approaches}).

Die instanziierten Layout-Patterns werden nach jeder Änderung des Diagramms durch einen Kontroll-Algorithmus ausgewertet, dessen Aufgabe es ist, alle Prädikate der Layout-Constraints zu erfüllen und damit ein valides Layout für das Diagramm zu berechnen.

Dieser Ansatz ist insbesondere für interaktive Umgebungen wie z.B. visuelle Editoren geeignet. Die Erstellung des Layouts wird dem Nutzer überlassen, indem er die Layout-Patterns instanziiert. Diese können entweder global oder für ausgewählten Knoten spezifiziert sein. Wenn keine Instanzen der Layout-Patterns verfügbar sind, können die Knoten im Diagramm frei positioniert werden. Anderseits wird die Interaktion durch die instanziierten Layout-Patterns beeinflusst und möglicherweise eingeschränkt.

Der beschriebene Ansatz wurde in Form eines Layout-Frameworks\footnote{\url{http://www.unibw.de/inf2/Personen/Wissen_Mitarbeiter/sonja/research/layoutframework}} implementiert und in visuellen Editoren eingesetzt, die mithilfe von \textit{DiaMeta}\footnote{Ein Framework zur Generierung von Editoren für visuelle Sprachen basierend auf Spezifikationen mittels Metamodelle. \url{http://www.unibw.de/inf2/DiaGen}} bzw. \textit{Graphical Editing Framework}\footnote{\url{http://www.eclipse.org/gef}} erzeugt wurden \cite{Maier12A-Pattern-based}.

In Abbildung \ref{fig:diameta-graph-editor-screenshot} wird ein Screenshot des \textit{DiaMeta Graph Editors} zur Erstellung von gerichteten Graphen gezeigt. Die rechte Sidebar enthält Buttons zur Instanziierung von Layout-Patterns. In der rechten unteren Ecke sind die durch den Nutzer erstellten Layout-Patterns aufgelistet. In dem mittleren Bereich des Fensters befindet sich der Canvas, der neben der Darstellung der Knoten und Kanten auch die einzelnen instanziierten Layout-Patterns visualisiert. In dem aufgeführten Beispiel wurden Layout-Patterns für die horizontale und vertikale Ausrichtung, gleichmäßige Verteilung und hierarchisches Layout verwendet. Für eine bessere Vorstellung der Interaktion im Editor sind die offiziellen Screencasts\footnote{\textit{DiaMeta Graph Editor}: \url{http://www.sonjamaier.de/dyndraw/screencasts/graphEditor.mov}\newline\textit{DiaMeta Ecore Editor}: \url{http://www.sonjamaier.de/dyndraw/screencasts/ecoreEditor.mov}} anzuschauen.

\begin{figure}[hbt]
    \centering
    \includegraphics[width=0.9\textwidth]{assets/diameta-graph-editor-screenshot}
    \newcommand{\captionvalue}{Ein Screenshot des \textit{DiaMeta Graph Editors}}
    \caption[\captionvalue]{\captionvalue{ }(Quelle: \url{http://www.unibw.de/inf2/Personen/Wissen_Mitarbeiter/sonja/research/layoutframework/graphEditor.png}, Aufruf: 20.09.2014)}
    \label{fig:diameta-graph-editor-screenshot}
\end{figure}

\subsection{Anwendungsspezifische Ansätze}

Die Struktur-basierten benutzergesteuerten Ansätze für das halbautomatische Layout, die in Abschnitt \ref{subsec:structure-based-user-controlled-approaches} beschrieben wurden, sind sehr universell und können in Editoren für verschiedene visuellen Sprachen eingesetzt werden. Aus diesem Grund können die syntaktischen und semantischen Eigenschaften der Sprachen nicht gründlich in den Algorithmen berücksichtigt werden. Dahingegen gibt es anwendungsspezifische Ansätze, deren Algorithmen für konkrete visuelle Sprachen entworfen sind. Diese werden im Folgenden vorgestellt.

\subsubsection{Smart Layout in MindNode}
\label{subsubsec:smart-layout-in-mindnode}

\textit{MindNode}\footnote{\url{https://mindnode.com}} ist eine benutzerfreundliche Desktop-Anwendung zur Erstellung von Mindmaps für \textit{Mac OS X}. Die Mindmaps haben eine baumbasierte Struktur mit einem oder mehreren zentralen Knoten. Weiterhin gilt, dass jeder Knoten mehrere Unterknoten besitzen kann und alle Knoten außer den zentralen Knoten genau einen Oberknoten haben. Die hierarchischen Relationen werden mit farbigen Zweigen dargestellt. Zusätzlich bietet \textit{MindNode} die Möglichkeit der Erstellung von Querverbindungen zwischen Knoten aus unterschiedlichen Teilen der Mindmap, die mit gestrichelten Pfeilen gekennzeichnet werden \cite{14MindNode}. In Abbildung \ref{fig:mindnode-example} ist ein Beispiel einer Mindmap zu sehen.

\begin{figure}[hbt]
    \centering
    \includegraphics[
        width=\textwidth,
        trim={0 1.5cm 0 1.5cm},
        clip
    ]{assets/mindnode-example}
    \caption{Beispiel einer Mindmap in \textit{MindNode}}
    \label{fig:mindnode-example}
\end{figure}

\textit{MindNode} stellt eine interaktive Funktion für eine halbautomatische Layout-Unterstützung namens \enquote{Smart Layout} bereit \cite{14MindNode}. Wenn diese Funktion ausgeschaltet ist, können einzelne Knoten der Mindmap frei positioniert werden. Dahingegen, wenn die Funktion eingeschaltet ist, nimmt die Mindmap eine vorgerechnete Baumstruktur an und die Verschiebungsaktion eines Knotens drückt in diesem Fall die Absicht einer Layout-Modifikation aus.

\begin{figure}[hbt]
    \newcommand{\subfigurewidth}{\textwidth}
    \newcommand{\graphicsscale}{0.25}
    \begin{subfigure}{\subfigurewidth}
        \centering
        \includegraphics[scale=\graphicsscale]{assets/mindnode-smart-layout-a}
        \caption{}
        \label{fig:mindnode-smart-layout-a}
    \end{subfigure}
    \begin{subfigure}{\subfigurewidth}
        \centering
        \includegraphics[scale=\graphicsscale]{assets/mindnode-smart-layout-b}
        \caption{}
        \label{fig:mindnode-smart-layout-b}
    \end{subfigure}
    \begin{subfigure}{\subfigurewidth}
        \centering
        \includegraphics[scale=\graphicsscale]{assets/mindnode-smart-layout-c}
        \caption{}
        \label{fig:mindnode-smart-layout-c}
    \end{subfigure}
    \caption{Die Verschiebungsaktion mit der eingeschalteten Funktion \enquote{Smart Layout} in \textit{MindNode}}
    \label{fig:mindnode-smart-layout}
\end{figure}

In Abbildung \ref{fig:mindnode-smart-layout} wird die Funktionsweise der Funktion \enquote{Smart Layout} anhand der Verschiebungsaktion gezeigt. Zunächst wird in \ref{fig:mindnode-smart-layout-a} ein Knoten angeklickt, dessen Position angepasst werden soll. Mit \enquote{Drag and Drop} wird in \ref{fig:mindnode-smart-layout-b} der Knoten auf die gewünschte Position verschoben. Die Verbindung zu dem Oberknoten wird dabei mit einem helleren Zweig veranschaulicht. Nach dem Loslassen der Maustaste in \ref{fig:mindnode-smart-layout-c} wird die gewünschte Position des verschobenen Knotens ausgewertet, ein neues Layout der Mindmap anhand des Hinweises durch die Verschiebungsaktion berechnet und anschließend auf die Mindmap angewendet. Dabei werden alle beeinflussten Knoten von ihren aktuellen Positionen zu ihren neuen Positionen mithilfe einer Animation bewegt. Somit kann der Nutzer das Layout beeinflussen, wird aber in den Möglichkeiten eingeschränkt. Die integrierte Layout-Berechnung führt zu einem optimalen Layout und ist vor allem durch die Struktur der visuellen Sprache für Mindmaps möglich.

% MindNode: Unterknoten des zentralen Knotens können auf die linke bzw. rechte Seite verschoben werden & initiales Layout

\subsection{Zusammenfassung der Eigenschaften}

\subsubsection{Vorteile}

\begin{itemize}

\item
\textbf{Unterstützung der interaktiven Bearbeitung}
Die Ansätze für das halbautomatische Layout sind für interaktive Umgebungen geeignet und ermöglichen eine unmittelbare Bearbeitung des Diagramms.

\item
\textbf{Persistente Strukturbeschreibung}
Die durch den Nutzer ausgeführten Änderung haben einen persistenten Charakter und werden während der weiteren Bearbeitung des Diagramms nicht verworfen. Außerdem erfolgt die Beschreibung der Struktur mithilfe von Regeln, die dem Nutzer zur Verfügung stehen. Dadurch führt die halbautomatische Layout-Unterstützung zur Vereinfachung der Layout-Erstellung.

\item
\textbf{Erhaltung des mentalen Modells}
Da die Wahrnehmungsorganisation in den interaktiven Umgebungen eine große Bedeutung hat \cite{ShieberKosak93Automating, Maier12A-Pattern-based}, zeichnen sich die halbautomatischen Ansätze durch die Erhaltung des mentalen Modells während der Ausführung von inkrementellen Änderungen aus \cite{GladischSchumann14Semi-Automatic}.

\end{itemize}

\subsubsection{Nachteile}

\begin{itemize}

\item
\textbf{Fehlende Einhaltung der ästhetischen Prinzipien}
Aufgrund der Überlassung der Strukturbeschreibung dem Nutzer können die ästhetischen Prinzipien verletzt werden. Insbesondere die allgemeinen Ansätze besitzen eine mangelhafte bis keine Unterstützung der syntaktischen und semantischen Regeln von konkreten Diagrammtypen.

\end{itemize}
