%%%%%%%%%%%
% Summary %
%%%%%%%%%%%

\chapter{Zusammenfassung}

\section{Fazit}

% In dieser Arbeit wurde kein vollständiger Ansatz gezeigt, sondern eine Richtung für eine mögliche Implementierung...
% kein allgemeiner Ansatz, bildet eine Grundlage und präsentiert mögliche Bedienungsmechanismen für die Entwicklung eines allgemeinen Ansatzes
% Dies wird der weiteren Forschung überlassen.
% basiert auf Einschränkungen der Möglichkeiten, die der Nutzer ausführen kann
% dadurch ist es möglich für konkrete Diagrammtypen spezielle Algorithmen einzusetzen

\section{Ausblick}

% Entwicklung eines allgemeinen Algorithmus
% Constraintlöser für Patterns, da bestehende Layout-Engines zu spezifisch -> soll allgemein sein und ein richtiger Constraintlöser sein
% Vereinfachung des Ansatzes von Sonja Maier, Erlauben dem Nutzer selbst Patterns zu erstellen (Kombinieren mit automatischen je nach Semantik)
% Historie der Layouts für weitere Berechnungen
% Aktionen im Prototypen (Auswahl, Löschen von ausgewählten Knoten/Kanten, Ändern der Knoten für eine Kante)
% Semantik, Erweitern für Klassendiagramme
% Multi-Touch
% Schwachstellen und Verbesserungspotential
% Low-Prio Elemente: Notizen (UML)
% Möglichkeit der Interaktivität der Patterns (Visualisierung, Erstellen durch den Nutzer)
% Einsatz eines Constraint-Solvers oder eines ausgereiften Algorithmus (wie Maier)
% Patterns-Metamodell -> Wiederverwendung für mehrere Diagramm-Typen
% Prototyp: Implementierung der impliziten Layout-Patterns (für den Constraint-Solver)
% Unterstützung für Cluster/Zonen
% weitere Eigenschaften der Mausbewegung wie Richtung/Geschwindigkeit
% Problem mit Verschiebung von mehreren Elementen gleichzeitig
% Unterstützung von sekundären Knoten für UML-Notizen
