%%%
% Interactive User-Controlled Layout
%%%

\section{Interaktives halbautomatisches Layout}
\label{sec:interactive-user-controlled-layout}

% Kombination der Vorteile der Ansätze aus beiden vorigen Gruppen
% Interaktion (manuelles Layout) + Berechnung und Berücksichtigen der ästhetischen Patterns (automatisches Layout) [Gladisch]
% nutzer-gesteuert
% Nutzer kann einige Aspekte des Layouts beeinflussen, z.B. durch direkte Positionierung der Knoten bzw. Kanten oder durch eine Form von Feedback [Arvo02]
% inkrementelles Erstellen -> automatische Ansätze nicht möglich, eher Sketching [Arvo02]
% ausgelegt für Änderungen [Gladisch, ...]

\subsection{Constraint-basierte Ansätze}

% Maier Chapter 2.3 !!
% Screenshot von Dunnart

\subsection{Pattern-basierter Ansatz}

% Link auf die Videos

\subsection{Anwendungsspezifische Ansätze}

\subsubsection{Smart Layout in MindNode}

% anwendungspezifisch

\subsubsection{EditLens}

% Multi-Touch Layout Techniques (Alignment Guides) TUD ?

\subsection{Eigenschaften und Vergleich}

% Wahrnehmungsorganisation hat eine größere Priorität als reine Berücksichtigung der syntaktischen Ästhetik [Shieber]
% Beides Kombinieren [Shieber]

% Nur visualisieren vs. auch Editieren [Gladisch]
% Nutzer entwickeln ein mentales Modell des Diagramms, das soll bei den inkrementellen Updates berücksichtigt werden [Gladisch]











