%%%
% Existing Approaches
%%%

\chapter{Bestehende Ansätze für Layout von Diagrammen}

Die bestehenden Ansätze für Layout von Diagrammen lassen sich in 3 Gruppen nach dem Grad der Interaktivität und der Art der Layout-Unterstützung unterteilen.

Zu einem gibt es das interaktive \textbf{manuelle Layout}, das auf der Freihand-Bearbeitung basiert und dem Nutzer ermöglicht, die Layout-Eigenschaften der Objekte im Diagramm wunschgemäß unter der Möglichkeit der Annahmen von Layout-Vorschlägen zu verändern. Das manuelle Layout wird in meisten graphischen Editoren eingesetzt und wird näher im Abschnitt \ref{sec:manual-layout} erläutert.

Weiterhin gibt es Ansätze für das \textbf{vollautomatische Layout}, die optimale Layouts unter Berücksichtigung von ästhetischen Prinzipien produzieren. Sie haben einen statischen Charakter und können von dem Nutzer nur minimal beeinflusst werden. Somit sind sie nicht für interaktive Umgebungen geeignet. Diese Ansätze werden im Abschnitt \ref{sec:automatic-layout} beschrieben.

Schließlich gibt es Ansätze für das \textbf{halbautomatische Layout}, die das Layout automatisiert berechnen, lassen aber den Nutzer das berechnete Layout interaktiv zu beeinflussen. Somit kombinieren sie die Vorteile der vorigen beiden Kategorien. Mit diesen Ansätzen beschäftigt sich der Abschnitt \ref{sec:interactive-semi-automatic-layout}.

%%%
% Manual Layout Functions
%%%

\section{Manuelles Layout}
\label{sec:manual-layout}

In Anwendungen zum Erstellen von Diagrammen wie z.B. Vektorgrafik-Software, Präsentationsprogramme oder CASE-Tools wird ausschließlich das manuelle Layout unterstützt, d.h. die einzelnen Bestandteile können im Diagramm frei positioniert werden. Somit kann der Nutzer ein gewünschtes Layout erreichen, muss dafür aber viele manuellen Schritte tätigen \cite{Eichelberger05Aesthetics}. Da sich der Inhalt des Diagramms während der Erstellung ständig ändert, können die manuellen Anpassungen des Layouts zur Frustration des Nutzers führen. Um einige Hürden des manuellen Layouts zu überwinden, bieten die Anwendungen eine Reihe an Funktionen, die dem Nutzer diese Tätigkeit bequemer machen.

Da das manuelle Layout auf direkter Interaktion des Nutzers mit dem Diagramm basiert, haben auch die Hilfefunktionen einen interaktiven Charakter. Außerdem weisen sie ein temporäres Verhalten auf, d.h. bei ihrer Nutzung werden keinerlei Regeln oder Constraints erstellt, wie das meistens bei den halbautomatischen Ansätzen der Fall ist (siehe Abschnitt \ref{sec:interactive-semi-automatic-layout}). Im Detail werden die Hilfefunktionen im Abschnitt \ref{subsec:help-functions-for-manual-layout} beschrieben.

% TODO: Namen besser formulieren
\subsection{Anwendungen}
\label{subsec:applications-for-manual-layout}

Im Folgenden werden drei Anwendungen vorgestellt, an den die Hilfefunktionen für das manuelle Layout gezeigt werden.

\subsubsection{OmniGraffle}
\label{subsubsec:omnigraffle}

Die von OmniGroup\footnote{\url{http://omnigroup.com}} entwickelte kommerzielle Anwendung OmniGraffle ist eins der einfachsten Tools zur Erstellung von Diagrammen unter Mac OS X \cite{Olsen10OmniGraffle}. OmniGraffle ist sehr flexibel und lässt sich für viele verschiedene Aufgaben einsetzen, z.B. von einem grafischen Entwurf einer Webseite bis zum Zeichnen von Klassendiagrammen. Das liegt vor allem daran, dass für die gezeichneten Diagramme kein semantisches Modell vorliegt und dass die Diagrammbestandteile als reine vektorgrafische Objekte repräsentiert werden. Detaillierte Information zu OmniGraffle können in \cite{08OmniGraffle}, \cite{Olsen10OmniGraffle} oder auf der offiziellen Webseite\footnote{\url{http://omnigroup.com/omnigraffle}} gefunden werden.

Die meisten im Abschnitt \ref{subsec:help-functions-for-manual-layout} beschriebenen Hilfefunktionen für das manuelle Layout werden anhand von OmniGraffle 5\footnote{Die aktuelle Version von OmniGraffle ist 6 (\url{http://www.omnigroup.com/blog/omnigraffle-6-is-here}), die Erkenntnisse für diese Arbeit wurden allerdings mit der Version 5 gewonnen.} gezeigt. OmniGraffle bietet auch eine Unterstützung für automatisches Layout, die näher im Abschnitt \ref{subsubsec:omnigraffle-auto-layout} beschrieben wird.

\subsubsection{Keynote}
\label{subsubsec:keynote}

Apple Keynote ist eine Anwendung zur Erstellung von Präsentationen für Mac OS X und iOS. Es bietet ähnliche Funktionen wie Microsoft PowerPoint\footnote{\url{http://office.microsoft.com/en-us/powerpoint/}} und bildet somit eine Alternative zu dem bekannten Präsentationsprogramm. Neben den Achsendiagrammen zur Visualisierung von Werten wie Linien- oder Balkendiagramme ermöglicht Keynote mit Hilfe von vektorgrafischen Objekten einfache graph-basierte Diagramme zu erstellen. Weitere Informationen sind in \cite{11Keynote} oder auf der offiziellen Webseite\footnote{\url{http://www.apple.com/de/mac/keynote/}} zu finden. 

Keynote besitzt ähnliche oder sogar identische Hilfefunktionen für das manuelle Layout wie OmniGraffle. Einige im Abschnitt \ref{subsec:help-functions-for-manual-layout} werden an Keynote 6 gezeigt. Das automatische Layout wird in Keynote nicht unterstützt.

\subsubsection{Visual Paradigm}
\label{subsubsec:visual-paradigm}

Visual Paradigm ist ein plattformunabhängiges CASE-Tool von der gleichnamigen Firma und unterstützt neben UML-Diagrammen auch weitere visuellen Sprachen \cite{14Visual}. Das Tool basiert auf einer Freihandbearbeitung und bietet neben Hilfefunktionen für das manuelle Layout (siehe Abschnitt \ref{subsec:help-functions-for-manual-layout}) auch Unterstützung für automatische Layout-Algorithmen (siehe Abschnitt \ref{subsec:visual-approaches}) \cite{Fuhrmann11On-the-Pragmatics}. Mit näheren Beschreibung von Visual Paradigm beschäftigen sich \cite{14Visual}, \cite[S.313-314]{Fuhrmann11On-the-Pragmatics} und die offizielle Webseite\footnote{\url{http://www.visual-paradigm.com}}.

% TODO: Bezug auf Anwendungen und Abschnitt 3.1.1 überprüfen
% TODO: Unterstützte Anwendungen aus Abschnitt 3.1.1 erwähnen
\subsection{Hilfefunktionen für manuelles Layout}
\label{subsec:help-functions-for-manual-layout}

\subsubsection{Raster}
\label{subsubsec:grid}

Viele Visualisierungsprogramme (unter anderem alle drei im Abschnitt \ref{subsec:applications-for-manual-layout} erwähnten) unterstützen die Aufteilung des Canvas in einen gleichmäßigen Raster, der durch Rasterlinien dargestellt werden kann. Diese Linien können zum Ausrichten von Objekten im Canvas verwendet werden, das vor allem für die Diagramme nützlich ist, in den die Größen der Objekte eine wichtige Rolle spielen wie z.B. Grundrisse \cite{08OmniGraffle, Olsen10OmniGraffle, 11Keynote, 14Visual}.

\paragraph{Snap-to-Grid}

Die reguläre Verschiebungsoperation von Objekten im Diagramm funktioniert in der Regel wie folgt: Der Nutzer wählt mit dem Mauszeiger ein Objekt aus und schiebt das Objekt mit gedrückter Maustaste auf seine neue Position. Wenn die Funktion \textit{Snap-to-Grid} nicht aktiviert ist, kann die Zielposition beliebig sein. Während der Verschiebung wird die Position des Objekts kontinuierlich angepasst, so dass der Nutzer einen visuellen Feedback bekommt. Dieser Vorgang ist in Abbildung X dargestellt.

% TODO: Abbildung: Verschiebungsoperation mit nicht aktivierter Funktion \textit{Snap-to-Grid} in OmniGraffle

Wenn die Funktion \textit{Snap-to-Grid} nun aktiviert ist, wird im Unterschied zu dem vorigen Vorgang die Zielposition so eingeschränkt, dass das verschobene Objekt an die Rasterlinien ausgerichtet wird. Der Mauszeiger kann dabei frei positioniert werden, das Objekt nimmt aber immer die nächste ausgerichtete Position an. Dieser Vorgang ist in Abbildung X dargestellt.

% TODO: Abbildung: Verschiebungsoperation mit aktivierter Funktion \textit{Snap-to-Grid} in OmniGraffle

Neben der Verschiebungsoperation wird auch die Operation der Größenänderung eingeschränkt, indem das manipulierte Objekt nur so eine Größe annehmen kann, die die Ausrichtung an die Rasterlinien nicht verletzt.

\paragraph{Ausrichten im Raster}

Das Ausrichten an die Rasterlinien kann auch nachträglich bewirkt werden. Dies passiert durch die Auswahl des zu ausgerichteten Objekts und dem Ausführen der Funktion \textit{Ausrichten im Raster}. Danach wird die Position und Größe des Objekts so angepasst, dass das Objekt an die Rasterlinien ausgerichtet wird (siehe Abbildung X).  

% TODO: Abbildung: Funktion \textit{Ausrichten im Raster} in OmniGraffle

\subsubsection{Ausrichten und Verteilen von Objekten in Relation zueinander}
\label{subsubsec:alignment-and-distribution}

Ähnlich wie der Raster (Abschnitt \ref{subsubsec:grid}) werden die Funktionen zum Ausrichten und Verteilen von ausgewählten Objekten häufig in Visualisierungsprogrammen eingesetzt. Sie funktionieren wie folgt: Der Nutzer wählt im Diagramm Objekte aus, an die die Funktion angewendet werden soll. Nachher wählt er eine konkrete Form der Funktion aus, die anschließend auf die ausgewählten Objekt angewendet wird \cite{11Keynote}.

\paragraph{Ausrichten}

Die ausgewählten Objekte können in Relation zueinander ausgerichtet werden. Dafür muss nach der Auswahl von mindestens 2 Objekten die Form der Ausrichtung angegeben werden. Die Objekte können entweder vertikal (linke Kanten, Mittelpunkte oder rechte Kanten) oder horizontal (obere Kanten, Mittelpunkte oder untere Kanten) ausgerichtet werden. Nach dem Ausführen dieser Funktion werden die Positionen der ausgewählten Objekte so angepasst, dass die gewünschte Ausrichtung erfüllt ist \cite{11Keynote, 08OmniGraffle}. Ein Beispiel ist in Abbildung X zu finden.

% TODO: Abbildung: Anwendung der horizontalen Ausrichtung in Keynote

\paragraph{Verteilen}

Wenn mindestens 3 Objekte ausgewählt werden, kann die Funktion zum gleichmäßigen Verteilen der Objekte angewendet werden. Dabei wird die Position der äußeren Objekte fixiert und die Position von umschlossenen Objekten wird so angepasst, dass die vertikalen bzw. horizontalen Abstände zwischen den Objekten gleich sind \cite{11Keynote}.

% TODO: Abbildung: Anwendung der horizontalen Verteilung in Keynote

\subsubsection{Smart Guides}
\label{subsubsec:smart-guides}

Eine sehr hilfreiche Funktion, die in den meisten Grafik- und Präsentationsprogrammen unterstützt wird, sind Hilfslinien für Ausrichtung und relative Positionierung (engl. \textit{Smart Guides}) \cite{11Keynote}. Wenn diese Funktion aktiviert ist, werden während der Manipulation des ausgewählten Objekts farbige Linien im Canvas eingeblendet. Diese Linien geben dem Nutzer einen visuellen Feedback zur geeigneten Ausrichtung des manipulierten Objekts relativ zu anderen Objekten im Diagramm. Außerdem wird ähnlich wie bei der Funktion \textit{Snap-to-Grid} die freie Manipulation eingeschränkt, das in der Regel zu besseren Layout-Ergebnissen führt. Die Hilfslinien werden nach der Beendigung der Operation wieder ausgeblendet und die Position bzw. Größe des manipulierten Objektes wird angepasst. Genauso wie auch bei allen anderen Hilfefunktionen für das manuelle Layout handelt es sich bei den Hilfslinien um eine temporäre Layout-Funktion. Somit wird die Ausrichtung der Objekte nur in den Eigenschaften der Objekte (Position und Größe) ausgedrückt und es wird keine explizite Regel erstellt.

\paragraph{Hilfslinien zur Ausrichtung (Alignment Guides)}

Die erste Art der Hilfslinien ist für die Ausrichtung des verschobenen Objekts an die Kanten bzw. Mittelachsen von anderen Objekten in Diagram nützlich. Die Linien werden während der Verschiebungsoperation eingeblendet, wenn das verschobene Objekt an ein anderes Objekt ausgerichtet wird \cite{11Keynote}. In Abbildung X wird ein Beispiel der Hilfslinien zur Ausrichtung der unteren Kanten von 2 Objekten während der Verschiebungsoperation gezeigt.

% TODO: Abbildung: Hilfslinien zur Ausrichtung während der Verschiebungsoperation in OmniGraffle

Aus der Abbildung X kann man entnehmen, dass sich um die untere Kante des inaktiven Objekts ein Bereich bildet, in dem das verschobene Objekt auf die ausgerichtete Position hinspringt. Dies passiert wenn die Gerade, die durch eine Kante oder Mittelachse des verschobenen Objektes gebildet wird, in der Nähe einer anderen Geraden, die durch eine Kante oder Mittelachse eines anderen Objektes im Diagramm gebildet wird, befindet und der Abstand unter einer bestimmten Schwelle liegt. Dieses Verhalten wird in Abbildung X visualisiert.

% TODO: Abbildung: Visualisierung des Bereichs für die Hilfslinien zur Ausrichtung

Solche Bereiche werden für alle Kanten und Mittelachsen von nicht ausgewählten Objekten in der horizontalen und vertikalen Richtung identifiziert. Wenn der Nutzer ein Objekt verschiebt und eine der Kanten oder die Mittelachse des verschobenen Objekts sich in einem der Bereichen befindet, wird das verschobene Objekt an die ausgerichtete Position positioniert, ohne den Mauszeiger zu beeinflussen. Das kann auch gleichzeitig für die horizontale und vertikale Richtung der Fall sein. Somit kann z.B. ein Objekt in einem anderen Objekt zentriert werden (siehe Abbildung \ref{fig:omnigraffle-alignment-guides-centering}).

\begin{figure}[hbt]
    \centering
    \includegraphics{resources/omnigraffle-alignment-guides-centering.png}
    \caption{Ausrichten eines Objekts in einem anderen Objekt mit Hilfe von einer horizontalen und einer vertikalen Hilfslinie zur Ausrichtung in OmniGraffle}
    \label{fig:omnigraffle-alignment-guides-centering}
\end{figure}

Eine Veranschaulichung dieser Funktion bietet in Form einer JavaScript-Anwendung das Open-Source Projekt \textit{Alignment-Guides}\footnote{Quellcode: \url{https://github.com/mrflix/Alignment-Guides}, Demo-Anwendung: \url{http://mrflix.github.io/Alignment-Guides/}}.

\paragraph{Abstandshilfslinien (Distance Guides)}

Die zweite Art der Hilfslinien dient dazu, die Abstände zwischen 3 Objekten in einer horizontalen oder vertikalen Reihe gleich zu halten. Während der Verschiebung eines der 3 Objekten wird ein Bereich gebildet, in dem das Objekt zu der Position hinspringt, in der die Abstände zwischen den einzelnen Objektpaaren gleich sind. Zusätzlich zu den Hilfslinien wird auch der Abstand eingeblendet \cite{11Keynote, Olsen10OmniGraffle}.

\begin{figure}[hbt]
    \centering
    \includegraphics{resources/omnigraffle-distance-guides.png}
    \caption{Abstandshilflinien in OmniGraffle}
    \label{fig:omnigraffle-distance-guides}
\end{figure}

\paragraph{Größenhilfslinien (Sizing Guides)}

Im Unterschied zu den zwei vorigen Arten der Hilfslinien wird die letzte Art durch die Operation der Größenänderung eines Objektes hervorgerufen. Wenn sich die Größe des manipulierten Objekts der Größe eines anderen Objekts im Diagramm nähert und die Differenz unter einer bestimmten Grenze liegt, wird die Größe des anderen Objekts übernommen. Dies funktioniert getrennt für die Breite und Höhe (siehe Abbildung \ref{fig:omnigraffle-sizing-guides}).

\begin{figure}[hbt]
    \centering
    \includegraphics{resources/omnigraffle-sizing-guides.png}
    \caption{Größenhilfslinien in OmniGraffle}
    \label{fig:omnigraffle-sizing-guides}
\end{figure}

\subsubsection{Manuelle Hilfslinien}

Weiterhin bieten OmniGraffle und Keynote die Funktion der Erstellung von manuellen Hilfslinien an \cite{08OmniGraffle, 11Keynote}. Diese Funktion ist sehr ähnlich zu \textit{Smart Guides}, unterscheidet sich aber wie folgt:

\begin{itemize}
    \item Die Hilfslinien werden vom Nutzer manuell platziert - entweder horizontal oder vertikal.
    \item Die Hilfslinien sind global für das gesamte Diagramm und damit nicht relativ zu einem anderen Objekt im Diagramm.
    \item Die Hilfslinien sind permanent sichtbar, auch wenn keine Verschiebungsoperation stattfindet.\footnote{Die Anzeige der manuellen Hilfslinien kann sowohl in OmniGraffle als auch in Keynote global aus- und eingeschaltet werden.}
\end{itemize}

Die Gemeinsamkeit besteht darin, dass das Ausrichten an die Hilfslinien identisch wie bei den \textit{Smart Guides} funktioniert (siehe Abschnitt \ref{subsubsec:smart-guides}). Ein Bespiel der manuellen Hilfslinie in OmniGraffle wird in Abbildung \ref{fig:omnigraffle-manual-guides} dargestellt.

\begin{figure}[hbt]
    \centering
    \includegraphics{resources/omnigraffle-manual-guides.png}
    \caption{Beispiel einer manuellen Hilfslinie in OmniGraffle}
    \label{fig:omnigraffle-manual-guides}
\end{figure}

\subsubsection{Gleiche Größe der Objekte}

Sehr trivial, aber dennoch nützlich ist die Funktion der Einstellung von gleichen Größen für ausgewählte Objekte. Nach der Auswahl von 2 oder mehreren Objekten im Diagramm hat der Nutzer die Möglichkeit, die Breite, die Höhe oder beide Dimensionen gleichzeitig auf die Werte des zuerst ausgewählten Objektes zu setzen (siehe Abbildung \ref{fig:omnigraffle-make-same-size}). Diese Funktion wird in OmniGraffle unterstützt \cite{Olsen10OmniGraffle}.

\begin{figure}[hbt]
    \centering
    \includegraphics{resources/omnigraffle-make-same-size.png}
    \caption{Anwendung der Funktion zur Einstellung von gleichen Größen in OmniGraffle}
    \label{fig:omnigraffle-make-same-size}
\end{figure}

\subsubsection{Verknüpfungspunkte für Verbindungen}

% TODO: Verweis auf die Grundlagen (graph-basierte Diagramme)

Die bisher diskutierten Hilfefunktionen haben sich ausschließlich mit der Positionierung und Bestimmung der Größe der Objekte auseinandergesetzt. Für die graph-basierten Diagramme sind neben der Knoten auch Kanten von Interesse, die in der Regel durch Verbindungslinien dargestellt werden. Wenn 2 Objekte mit einer Verbindungslinie verbunden werden, bleiben sie auch dann verbunden, wenn sich die Position oder Größe der Objekte ändert \cite{11Keynote}. In vielen Programmen (unter anderem in Keynote und OmniGraffle) werden die Verknüpfungspunkte der Verbindungslinie standardmäßig an die Stellen positioniert, an den eine gedachte Strecke, die die Mittelpunkte beider Objekte verbindet, die Kanten der Objekte schneidet (siehe Abbildung \ref{fig:omnigraffle-default-connection-points}) \cite{08OmniGraffle}.

\begin{figure}[hbt]
    \centering
    \includegraphics{resources/omnigraffle-default-connection-points.png}
    \caption{Verknüpfungspunkte einer Verbindung in OmniGraffle (links) mit Darstellung der Berechnung (rechts)}
    \label{fig:omnigraffle-default-connection-points}
\end{figure}

OmniGraffle bietet die Möglichkeit, mit Hilfe des Magnetwerkzeugs manuell die Verknüpfungspunkte zu definieren. Dies funktioniert auf zwei verschiedene Arten. Entweder kann der Nutzer die Verknüpfungspunkte beliebig in dem Objekt positionieren oder er wählt eine vordefinierte Anordnung aus. Dazu gehören u.a. die gleichmäßige Verteilung einer festen Anzahl an Verknüpfungspunkten entlang allen Kanten oder die Positionierung eines Verknüpfungspunktes an jedem Eckpunkt. Ein Beispiel der Darstellung von Verknüpfungspunkten ist in Abbildung \ref{fig:omnigraffle-magnets-example} zu finden.

\begin{figure}[hbt]
    \centering
    \includegraphics{resources/omnigraffle-magnets-example.png}
    \caption{Beispiel der manuellen Verknüpfungspunkte in OmniGraffle: Magnete an den Eckpunkten (links), 3 Magnete pro Kante (Mitte), ein manuell positionierter Magnet (rechts)}
    \label{fig:omnigraffle-magnets-example}
\end{figure}

Sobald mindestens ein Magnet dem Objekt hinzugefügt wurde, wird nicht mehr der Verknüpfungspunkt mit Hilfe des Mittelpunktes berechnet wie oben beschrieben, sondern die Verbindungslinie wird immer von dem nächsten Verknüpfungspunkt angezogen (siehe Abbildung \ref{fig:omnigraffle-magnets-example}). Alternativ kann der Nutzer ein Ende der Verbindungslinie an einen konkreten Verknüpfungspunkt binden. In dem Fall bleibt die Verbindungslinie an den gewählten Verknüpfungspunkt gebunden, auch wenn ein anderer Verknüpfungspunkt näher ist.

\subsection{Eigenschaften und Vergleich}

% Mit dieser Funktion (manuelle Hilfslinien) ist der Nutzer in der Lage das Layout präziser zu steuern.
% keine Unterstützung für Diagrammtypen
% Klassenelement = Zusammenbauen von mehreren Rechtecken und Textfeldern
% kein Metamodell hinter der graphischen Notation
% ohne jegliche semantische Unter
% Struktur und Semantik wird von den Vektorgrafikprogrammen (Zeichentools) nicht berücksichtigt
% Struktur- und Semantikregeln
% die Layout-Operationen sind nicht persistent
% mögliche Verletzung durch manuelle Aktionen
% Unterscheidung zwischen CASE-Tools und anderen Diagrammprogrammen
% interaktiv
% temporär -> nicht persistent
% nur Tools -> der Nutzer muss die ästhetischen Prinzipien "selber einhalten"

% Vorteile:
%% ermöglicht eine direkte Interaktion des Nutzers mit dem Diagramm
%% flexibel -> Nutzer kann viel beeinflussen
%% intuitiv

% Nachteile:
%% viel an manueller Arbeit

%großen Einfluss auf das resultierende Layout des Diagramms hat
% der Nutzer muss sich mit dem Layout viel auseinandersetzen -> Inhalt der Diagramme soll im Vordergrund stehen
% keine semantische Unterstützung

%%%
% Automatic Layout
%%%

\section{Automatisches Layout}
\label{sec:automatic-layout}

% TODO: Verweis auf "graph-basierte Diagramme" in Begriffsklärung
Den graph-basierten Softwarediagrammen unterliegt die abstrakte Struktur des Graphen (siehe Abschnitt X). Mit der graphischen Darstellung von Graphen beschäftigt sich die mathematische Disziplin des Graphzeichnens. Ihre Aufgabe besteht darin, Layout-Algorithmen zu entwerfen, die optimale Layouts in Hinsicht auf ästhetische Regeln erzeugen, d.h. die Layout-Eigenschaften wie Position der Knoten und Routen der Kanten berechnen \cite{Eichelberger05Aesthetics, Arvo02Techniques, Siebenhaller03Automatisches, Maier12A-Pattern-based}.

Unter dem automatischen Layout für graph-basierte Diagramme sind daher vollautomatische Algorithmen zu verstehen, die für ein gegebenes Diagramm die optimalen Layout-Eigenschaften berechnen und das Diagramm entsprechend anpassen \cite{Fuhrmann11On-the-Pragmatics}.

Um während der Erstellung des Diagramms immer ein optimales Layout zu erhalten, muss der Layout-Algorithmus kontinuierlich nach jeder Änderung aufgerufen werden. Dies kann entweder automatisch oder manuell passieren.

Eine weitere Eigenschaft der Ansätze für das automatische Layout ist ein geringer Einfluss des Nutzers auf das Ergebnis des Layout-Prozesses. Oft können die Parameter des Algorithmus angepasst werden. Wenn aber die Interaktion mit dem resultierenden Diagramm unterstützt wird, hat sie keinerlei Einwirkung auf den Algorithmus und wird bei dem nächsten Aufruf des Algorithmus nicht beachtet.

Die Ansätze lassen sich in zwei Kategorien nach der Art der Sprache unterteilen, die als Eingabe für den Layout-Algorithmus verwendet wird\footnote{In der Literatur werden Ansätze für das automatische Layout meistens nach den Layout-Algorithmen kategorisiert. In dieser Arbeit richtet sich die Kategorisierung danach, wie die Ansätze zu bedienen sind und welche Art der Interaktion sie unterstützen.}. Zu einem aus einer Beschreibung eines Diagramms in einer \textbf{textuellen Sprache} unter Anwendung des Layout-Algorithmus eine graphische Repräsentation erzeugen. Weiterhin gibt es Ansätze, die auf Diagramme angewendet werden, die in einer \textbf{visuellen Sprache} modelliert sind. Diese Ansätze verändern die Layout-Eigenschaften der Diagrammelementen direkt. Beide Kategorien werden im Folgenden behandelt.

\subsection{Textbasierte Ansätze}

Die textbasierten Ansätze für das automatische Layout erfordern als Eingabe eine textuelle Beschreibung des Diagramms, die meistens in einer domänenspezifischen Sprache formuliert wird. Diese Beschreibung wird eingelesen und intern in ein abstraktes Modell umgewandelt, auf das der Layout-Algorithmus angewendet wird. Als Ausgabe wird eine statische Repräsentation des Diagramms in Form eines Bilds geliefert, die jegliche Möglichkeit der Interaktion vermisst. Durch die Entkopplung der Eingabe von der Ausgabe lässt sich das Diagramm nur durch eine Änderung des Quelltexts und einen wiederholten Aufruf des Layout-Algorithmus ändern.

% Live-Preview vs. Kommandozeilen-Tools

\subsubsection{Graphviz}

\textit{Beschreibung von Graphviz. Sprache Dot, Layout-Algorithmen, Kommandozeilen-Tools, Bibliothek, 3rd Party Web- und Desktop-Apps.}

% [Fuhrmann]
% Algorithmen aufzählen?
% http://webgraphviz.com

\subsubsection{Textbasierte UML-Tools}

\textit{Beschreibung von textbasierten UML-Tools.}

\paragraph{PlantUML}

\textit{Beschreibung von PlantUML.}

\paragraph{yUML}

\textit{Beschreibung von yUML.}

% [Fuhrmann]

\subsection{Visuelle Ansätze}
\label{subsec:visual-approaches}

Die visuellen Ansätze für das automatische Layout unterscheiden sich von den textbasierten darin, dass der Layout-Algorithmus auf eine visuelle Sprache angewendet wird.

% Eingabe == Ausgabe
% Ausgabe interaktiv, geänderte Stellen werden aber beim nächsten Aufruf nicht beachtet => kein direkter Einfluss auf das Layout
% unmittelbare Bearbeitung des Diagramms
% in der Regel manuelle Ausführung

% in Diagramm-Editoren eingebaut

% OmniGraffle: Graphviz Layout Engine
% Zusatzfunktion in OmniGraffle und CASE-Tools (Visual Paradigm anschauen)

% Unterstützung des automatischen Layouts in OmniGraffle
% Interaktion mit dem Diagramm wird von dem Layout-Algorithmus nicht berücksichtigt => Zerstören des mentalen Modells

\subsection{Eigenschaften und Vergleich}

%%%
% Interactive Semi-automatic Layout
%%%

\section{Interaktives halbautomatisches Layout}
\label{sec:interactive-semi-automatic-layout}

Wie bereits in diesem Kapitel präsentiert wurde, unterstützen die meisten Editoren zur Erstellung von Diagrammen Hilfefunktionen für das manuelle Layout (Abschnitt \ref{sec:manual-layout}) und eventuell auch automatische Layout-Algorithmen (Abschnitt \ref{sec:automatic-layout}). Das manuelle Layout ist sehr intuitiv, zeichnet sich durch die direkte Interaktion des Nutzers mit dem Diagramm aus, vermisst aber eine automatisierte Layout-Berechnung und Berücksichtigung der ästhetischen Prinzipien. Dieses Problem wird durch das automatische Layout bekämpft, das aber in dem Bereich der Interaktivität versagt \cite{GladischSchumann14Semi-Automatic}. Die Vorteile der Ansätze aus beiden genannten Kategorien lassen sich kombinieren und bilden eine neue Kategorie der Ansätze für die halbautomatische Layout-Unterstützung.

Diese Ansätze sind für Nutzer-bedingte Änderungen ausgelegt und ermöglichen eine inkrementelle Erstellung der Diagramme mit Bezug auf das Layout. Somit sind sie insbesondere für interaktive Umgebungen geeignet \cite{Arvo02Techniques, GladischSchumann14Semi-Automatic, Wybrow08Using}.


% dynamische Algorithmen, Sequenzen von Modifizierungen
% Nutzer kann einige Aspekte des Layouts beeinflussen, z.B. durch direkte Positionierung der Knoten bzw. Kanten oder durch eine Form von Feedback [Arvo02]

\subsection{Struktur-basierte Nutzer-gesteuerte Ansätze}

Die erste Kategorie der Ansätze für das interaktive halbautomatische Layout bilden die Struktur-basierten Nutzer-gesteuerten Ansätze, die dem Nutzer ermöglichen, das Layout des Diagramms durch Erstellung und Verwaltung von Strukturregeln\footnote{Ein Beispiel für eine solche Strukturregel ist die im Abschnitt \ref{subsubsec:alignment-and-distribution} beschriebene gleichmäßige Verteilung der ausgewählten Objekte in Relation zueinander.} zu beeinflussen. Diese Strukturregeln erinnern an die Hilfefunktionen für das manuelle Layout (siehe Abschnitt \ref{subsec:help-functions-for-manual-layout}), sind aber dahingegen persistent \cite{Wybrow08Using}. Sie werden bei der Berechnung des Layouts durch einen dynamischen Layout-Algorithmus berücksichtigt und eingehalten.

Die Struktur-basierten Nutzer-gesteuerten Ansätze lassen sich in zwei Gruppen unterteilen: in Constraint-basierte und Pattern-basierte Ansätze. Beide Gruppen werden im Folgenden vorgestellt.

\subsubsection{Constraint-basierte Ansätze}
\label{subsubsec:constraint-based-approaches}

Die Strukturregeln können mit Hilfe von Constraints umgesetzt werden. Ein valides Layout wird dadurch berechnet, in dem alle Constraints im Diagramm mit einem Constraintlöser ausgewertet werden.



% deklarativ
% welche Regeln sollen gelten anstatt wie
% Constraintlöser, verschiedene Typen (beschrieben in [Mai])

% Cassowary for layout of UI in Cocoa/iOS

% Dunnart: Screenshot, http://dunnart.org
% Nutzer kann Constraints erstellen und auf Teile des Diagramms anwenden

% Probleme mit der Performance, da der Algorithmus nach jeder kleinen Änderung laufen muss

\subsubsection{Pattern-basierter Ansatz}

Der in \cite{Maier12A-Pattern-based} und \cite{MaierMinas10Combination} präsentierte Pattern-basierte Ansatz für das Layout von Diagrammen drückt die Strukturregeln in Form von Layout-Patterns aus. Der Ansatz kann für beliebige visuellen Sprachen eingesetzt werden, die mit Meta-Modellen beschrieben werden können. Dabei besteht das sprachenspezifische Meta-Modell aus zwei Teilen, nämlich der Meta-Modelle für die abstrakte und konkrete Syntax. Das zuletzt genannte Meta-Modell beschreibt die visuellen Eigenschaften der Sprache und wird daher durch die Berechnung des Layouts beeinflusst.

Die Beschreibung der Layout-Patterns erfolgt allerdings nicht mit Hilfe der sprachenspezifischen Meta-Modellen, sondern mit sprachenunabhängigen Pattern-spezifischen Meta-Model\-len\footnote{In \cite{Maier12A-Pattern-based} werden u.a. Pattern-spezifische Meta-Modelle für Mengen von Elementen, geordnete Liste von Elementen oder Graphen vorgestellt.}. Bei der Instanziierung eines Layout-Patterns wird das sprachenspezifische Modell auf ein Pattern-spezifisches Modell abgebildet\footnote{Dieses Sachverhalt wird mit Hilfe eines Beispiels in \cite[S.59ff]{Maier12A-Pattern-based} anschaulich gemacht.}. Dadurch wird eine mögliche Wiederverwendung der Layout-Patterns für diverse visuellen Sprachen gewährleistet. Weiterhin zeichnet sich der Pattern-basierte Ansatz dadurch aus, dass die Layout-Patterns verschiedene Ansätze zum Layout von Diagrammen u.a. Algorithmen zum Graphenzeichnen (siehe Abschnitt \ref{subsubsec:graph-drawing-tools}) und Constraints-basierte Algorithmen (siehe Abschnitt \ref{subsubsec:constraint-based-approaches}) kombinieren und sich zu Nutze machen.


% Auflistung der Layout-Patterns: tree layout, layered layout, equal distance, alignment…

%Kontroll-Algorithmus läuft im Hintergrund nach jeder Änderung des Diagramms bzw. der Instanziierung eines Layout-Patterns

%nutzer-gesteuert, nicht eingeschränkt in Interaktion

%Wie bereits erwähnt, ist dieser Ansatz für interaktive


% Videos: http://www.sonjamaier.de/dyndraw/screencasts/graphEditor.mov & http://www.sonjamaier.de/dyndraw/screencasts/ecoreEditor.mov

Dieser Ansatz wurde in Form eines Layout-Frameworks\footnote{\url{http://www.unibw.de/inf2/Personen/Wissen_Mitarbeiter/sonja/research/layoutframework}} implementiert, das bisher noch nicht veröffentlicht wurde. Dennoch wurde es in visuellen Editoren eingesetzt, die mit Hilfe von DiaMeta\footnote{Ein Framework zur Generierung von Editoren für visuelle Sprachen basierend auf Spezifikationen mittels Meta-Modelle. \url{http://www.unibw.de/inf2/DiaGen/}} bzw. Graphical Editing Framework\footnote{\url{http://www.eclipse.org/gef/}} erzeugt wurden \cite{Maier12A-Pattern-based}.

\subsection{Anwendungsspezifische Ansätze}

\subsubsection{Smart Layout in MindNode}

% anwendungspezifisch
% Mindmaps 
% Screenshot der Funktion
% Möglich, weil spezifisch für die visuelle Sprache für Mindmaps (Bäume)

\subsubsection{EditLens}

% Multi-Touch Layout Techniques (Alignment Guides) TUD ?

\subsection{Eigenschaften und Vergleich}

% Wahrnehmungsorganisation hat eine größere Priorität als reine Berücksichtigung der syntaktischen Ästhetik [Shieber]
% Beides Kombinieren [Shieber]

% Nur visualisieren vs. auch Editieren [Gladisch]
% Nutzer entwickeln ein mentales Modell des Diagramms, das soll bei den inkrementellen Updates berücksichtigt werden [Gladisch]

% Der Nutzer kann das Diagramm unmittelbar bearbeiten
% das mentale Modell bleibt beibehalten [preserving the mental map (in dynamic context) instead of aesthetics (in static context)]
% ästhetische Regeln werden nicht eingehalten









% Tabelle für den Vergleich zwischen den bestehenden Ansätzen und meinem Ansatz
