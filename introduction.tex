%%%
% Introduction
%%%

\chapter{Einleitung}

% Agile Modeling
% Papier/Whiteboard vs. Zeichentools vs. CASE-Tools

% Zielstellung (interaktiv -> Visual Language Editors)

% Klassendiagramme meistens weniger als 10 Klassen (Graph zitieren + Rat von Scott Ambler)

%Motivation
%	agile Softwareentwicklung → Änderungen → MDD spielt keine Rolle (Desktop, iOS, mobil, web) → Skizzen → Agile Modellierung → Papier/Whiteboard vs. Software → Vorteile/Nachteile
%	Agile Modeling
%		fehlt in CASE-Tools
%		entweder wirklich Zeichnen bzw. komplexe mächtige Tools
%	Informationen aus dem EXIST-Antrag
%	Warum eingeschränkt?
%	Warum nutzergesteuert und nicht automatisch?
%Begriffsklärung “Layout”, “graph-basierte Softwarediagramme”, “interaktiv” (“diagrammspezifisch”)

%Papier vs. Software
%	Vorteile/Nachteile
%im Desktop-Bereich (Mauszeiger & Fenster - keine Multi-Touch Ansätze)
%Struktur der Arbeit
%Beispiele und Apps: Mac OS X

% Agile Modeling
% UML & MDD -> agile Softwareentwicklung

% Aus dem Antrag:
%Agile Softwareentwicklung, die sich nach den Grundsätzen aus dem Agile Manifesto4 richtet, freut sich heutzutage einer großen Beliebtheit. Der Grund dafür ist, dass sie im Vergleich zu den traditionellen Soft- wareentwicklungsprozessen (wie z.B. Unified Process oder Wasserfallmodell) schlanker und wesentlich flexibler ist. Desweiteren steht der Code im Fokus, das durch zahlreiche Praktiken wie z.B. Code-Refac- toring und Code-Reviews unterstützt wird.
%Der agile Ansatz wurde auch in die Softwaremodellierung übertragen und in der Methodik Agile Model- ing5 veröffentlicht. Diese auf Praktiken basierte Methodik wurde von Scott Ambler entwickelt und ver- sucht einen effektiven Weg für die Softwaremodellierung zu finden.
%Die Konzepte aus Agile Modeling sind an sich nicht revolutionär. Weil wir bisher noch keine Model- lierungssoftware gefunden haben, die diese Konzepte umsetzt, sehen wir auch an dieser Stelle ein Po- tential für Innovation. Somit wird Cases z.B. eine Archivierungsfunktion besitzen (als Unterstützung der Praktik “Discard Temporary Models”) oder nur eine hilfreiche Teilmenge der UML-Notation anbieten (als Unterstützung der Praktik “Depict Models Simply”).
